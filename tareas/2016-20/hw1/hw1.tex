
%--------------------------------------------------------------------
%--------------------------------------------------------------------
% Formato para los talleres del curso de Métodos Computacionales
% Universidad de los Andes
% 2015, curso de vacaciones
%--------------------------------------------------------------------
%--------------------------------------------------------------------

\documentclass[11pt,letterpaper]{exam}
\usepackage[utf8]{inputenc}
\usepackage[spanish]{babel}
\usepackage{graphicx}
\usepackage{tabularx}
\usepackage[absolute]{textpos} % Para poner una imagen en posiciones arbitrarias
\usepackage{multirow}
\usepackage{float}
\usepackage{hyperref}
\usepackage{breakurl}
\decimalpoint

\begin{document}
\begin{center}

\includegraphics[width=16cm]{header.png}

\vspace{1.0cm}
{\Large Herramientas Computacionales - Tarea 1} \\
\textsc{Semana 3 - Comandos b\'asicos de UNIX}\\
2016-II\\
\end{center}

%\begin{textblock*}{40mm}(10mm,20mm)
%  \includegraphics[width=3cm]{logoUniandes.png}
%\end{textblock*}

%\begin{textblock*}{40mm}(164mm,20mm)
%  \includegraphics[width=3cm]{logoUniandes.png}
%\end{textblock*}

\vspace{0.5cm}

\noindent
Los dos archivos c\'odigo fuente deben subirse a Sicua plus en un \'unico archivo
\verb".zip" con el nombre del estudiante en el formato \verb"NombreApellido_hw1.zip" antes que termine la clase.

\vspace{0.5cm}

\begin{questions}
 
\question[5.0] {\bf{Pilos de Matem\'aticas}}

Escriba un script llamado \verb"matematicas.sh" que primero descargue el archivo de \burlalt{notas.}{https://raw.githubusercontent.com/ComputoCienciasUniandes/HerramientasComputacionalesDatos/master/Homework/hw1/01_notas.tsv},
y genere un archivo (\verb"pasaron.txt") con el número de estudiantes de matemáticas que pasaron la materia.
Es válido pero no necesario generar otros archivos auxiliares para pasos intermedios.


\end{questions}

\end{document}
