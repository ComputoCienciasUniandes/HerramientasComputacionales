\documentclass[11pt,letterpaper]{exam}
\usepackage{amsmath}
\usepackage[utf8]{inputenc}
\usepackage[spanish]{babel}
\usepackage{graphicx}
\usepackage{tabularx}
\usepackage[absolute]{textpos} % Para poner una imagen en posiciones arbitrarias
\usepackage{multirow}
\usepackage{float}
\usepackage{hyperref}
\usepackage{breakurl}
\decimalpoint

\begin{document}
\begin{center}

\includegraphics[width=16cm]{header.png}

\vspace{1.0cm}
{\Large Herramientas Computacionales - Tarea 11}\\
2016-II
\end{center}

%\begin{textblock*}{40mm}(10mm,20mm)
%  \includegraphics[width=3cm]{logoUniandes.png}
%\end{textblock*}

%\begin{textblock*}{40mm}(164mm,20mm)
%  \includegraphics[width=3cm]{logoUniandes.png}
%\end{textblock*}

\vspace{0.5cm}

\noindent
El c\'odigo fuente debe subirse a Sicua plus en un \'unico archivo
\verb".ipynb" con el nombre del estudiante en el formato \verb"NombreApellido_hw11.ipynb" antes que termine la clase.

\vspace{0.5cm}

Escriba un notebook de ipython con los siguientes pasos

\begin{questions}

\question[0.5]\textbf{Suma de n\'umeros aleatorios}

Una funci\'on que calcule y retorne la suma de \verb'n' n\'umeros aleatorios, cada uno distribu\'ido uniformemente entre $0$ y $1$. Cada uno de estos n\'umeros se generan utilizando la funci\'on \verb'np.random.random()'. El n\'umero \verb'n' debe ser par\'ametro de la funci\'on.

\question[2.0]\textbf{10000 sumas de n\'umeros aleatorios - Histogramas}

Una funci\'on que corra \verb'10000' veces la funci\'on anterior y vaya almacenando los resultados en un arreglo. Esta funci\'on recibe como par\'ametro el n\'umero \verb'n' de elementos a sumar en cada una de las \verb'10000' iteraciones.

La funci\'on debe graficar un histograma del arreglo de sumas utilizando $50$ bins, y retornar dicho arreglo. 

\question[1.5]\textbf{Estad\'isticas}

Ejecute la funci\'on anterior para \verb'n = 50, 100, 200, 400' de tal forma que se muestren los $4$ histogramas. Para cada una de las ejecuciones calcule el promedio y la desviaci\'on est\'andar y realice un \verb'scatter' de desviaci\'on est\'andar (eje $y$) vs. promedio (eje $x$) que muestre los $4$ puntos correspondientes.

\question[1.0]\textbf{Ajuste}

Realice un fit de los puntos anteriores a la función $y = a\sqrt{x}$ utilizando \verb'curve_fit',  y grafique el fit junto con la gr\'afica realizada en el punto anterior.

\end{questions}

\end{document}
