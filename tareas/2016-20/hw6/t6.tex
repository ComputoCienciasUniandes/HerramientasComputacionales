\documentclass[11pt,letterpaper]{exam}
\usepackage{amsmath}
\usepackage[utf8]{inputenc}
\usepackage[spanish]{babel}
\usepackage{graphicx}
\usepackage{tabularx}
\usepackage[absolute]{textpos} % Para poner una imagen en posiciones arbitrarias
\usepackage{multirow}
\usepackage{float}
\usepackage{hyperref}
\usepackage{breakurl}
\decimalpoint

\begin{document}
\begin{center}

\includegraphics[width=16cm]{header.png}

\vspace{1.0cm}
{\Large Herramientas Computacionales - Tarea 6}
2016-II
\end{center}

%\begin{textblock*}{40mm}(10mm,20mm)
%  \includegraphics[width=3cm]{logoUniandes.png}
%\end{textblock*}

%\begin{textblock*}{40mm}(164mm,20mm)
%  \includegraphics[width=3cm]{logoUniandes.png}
%\end{textblock*}

\vspace{0.5cm}

\noindent
Los archivo del c\'odigo fuente debe subirse a Sicua plus en un \'unico archivo
\verb".zip" con el nombre del estudiante en el formato \verb"NombreApellido_hw6.zip" antes que termine la clase.

El objetivo de este ejercicio es modificar la clase \verb"Balon" introducida en el video para que incluya tambi\'en la direcci\'on horizontal y calcule la trayectoria en un tiro parab\'olico.

\vspace{0.5cm}

\begin{questions}
 
\question[2.5] {\bf{Implementaci\'on de la clase \verb"Balon"}}

Escriba un script de Python (\verb'.py') o un notebook de Jupyter (\verb".ipynb") donde implemente la clase \verb"Balon" con las siguientes adiciones

\begin{itemize}
\item El constructor \verb'__init__' recibe como par\'ametros \verb'x0, vx0, y0, vy0, m0': las posiciones y velocidades iniciales del bal\'on, y su masa.

\item Los atributos son las posiciones y velocidades actuales \verb'x, vx, y, vy', la masa \verb'm' y las \textbf{listas} de tiempos, posiciones y velocidades \verb'T, X, Vx, Y, Vy'.

\item Los atributos mencionados deben ser correctamente inicializados en el constructor. All\'i mismo debe ser asignado el primer elemento de \verb'T' a \verb'0', y los primeros elementos de las dem\'as listas seg\'un las condiciones iniciales.

\item El m\'etodo \verb'muevete' debe incluir tambi\'en la evoluci\'on de la posici\'on horizontal \verb'x' que se realiza de manera casi id\'entica a la evoluci\'on de \verb'y'.

\item El m\'etodo \verb'imprime' ahora se llamar\'a \verb'guarda' y ahora debe guardar los valores de \verb't, x, vx, y, vy' en las respectivas listas. Puede usar la funci\'on \verb'append' para hacerlo.
\end{itemize}
\question[1.5] {\bf{Creaci\'on del objeto y evoluci\'on}}

En un notebook de Jupyter, cree un objeto de la clase \verb'Balon', y realize la evoluci\'on similarmente a como se hizo en el video de preparaci\'on para un \verb'Deltat = 0.01' hasta un tiempo final de \verb'2.0'. Luego del ciclo las listas deben contener todos los valores de tiempos, posiciones y velocidades en el intervalo de tiempo considerado.

\question[1.0] {\bf{Gr\'afica}}

Realice una gr\'afica de $Y$ contra $X$ utilizando la misma sintaxis del video de preparaci\'on de tal forma que se vea la trayectoria parab\'olica del bal\'on.

No olvide escribir \verb'%pylab inline' al comienzo de su notebook para visualizar.

\end{questions}

\end{document}
