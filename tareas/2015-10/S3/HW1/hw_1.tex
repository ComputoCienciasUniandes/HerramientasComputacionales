
%--------------------------------------------------------------------
%--------------------------------------------------------------------
% Formato para los talleres del curso de Herramientas Computacionales
% Universidad de los Andes
% 2015-10
%--------------------------------------------------------------------
%--------------------------------------------------------------------

\documentclass[11pt,letterpaper]{exam}
\usepackage[utf8]{inputenc}
\usepackage[spanish]{babel}
\usepackage{graphicx}
\usepackage{mdframed}
\usepackage{tabularx}
\usepackage[absolute]{textpos} % Para poner una imagen completa en la portada
\usepackage{multirow}
\mdfdefinestyle{mystyle}{leftmargin=1cm,rightmargin=1cm,linecolor=red}
\usepackage{float}
\usepackage{hyperref}
\decimalpoint
%\usepackage{pst-barcode}
%\usepackage{auto-pst-pdf}

\newcommand{\base}[1]{\underline{\hspace{#1}}}
\boxedpoints
\pointname{ pt}
%\extrawidth{0.75in}
%\extrafootheight{-0.5in}
\extraheadheight{-0.15in}
%\pagestyle{head}

%\noprintanswers
%\printanswers
\renewcommand{\solutiontitle}{}
\SolutionEmphasis{\color{blue}}

\usepackage{upquote,textcomp}
\newcommand\upquote[1]{\textquotesingle#1\textquotesingle} % To fix straight quotes in verbatim

\begin{document}
\begin{center}
{\Large Herramientas Computacionales} \\
Taller 1 - \textsc{Unix}: línea de comandos \\
Fecha de publicación: {\small \it enero 28 de 2015}\\
\end{center}

\begin{textblock*}{40mm}(10mm,20mm)
  \includegraphics[width=3cm]{logoUniandes.png}
\end{textblock*}

\begin{textblock*}{40mm}(161mm,20mm)
  \includegraphics[width=3cm]{logoUniandes.png}
\end{textblock*}

\vspace{0.5cm}

{\Large Fecha de entrega:  \bf 3 de febrero antes de las 23:59 PM}

\vspace{0.5cm}

{\Large Instrucciones de Entrega}\\

Esta tarea se debe entregar a través de Sicua plus. Un solo script debe resolver todos los puntos requeridos, y el nombre del script debe llevar su nombre y apellido en el formato \verb"NombreApellido_HW1.sh". \\

En cada parte del ejercicio se entrega 1/3  de los puntos si el código propuesto es razonable, 1/3 si se puede ejecutar y 1/3 si entrega resultados correctos. El script debe llevar comentarios suficientes.


\vspace{0.5cm}


\begin{questions}

\question[100] {\bf Catálogo estelar}

El catálogo HYG contiene información astronómica de estrellas pertenecientes a 
la Vía Láctea. Nuestro ejercicio consiste en procesar este archivo usando algunas de las herramientas de la terminal (\verb+wget, tar, wc, awk, sed, grep, echo, $>$, $>>$+). El archivo se puede descargar de: \url{https://github.com/spsaaibi/ComputationalToolsData/raw/master/data/HYG/hyg.tar.gz}\\


\begin{parts}
	\part[15] Descargar y descomprimir el archivo \verb+hyg.tar.gz+. El resultado debe ser \verb+hyg.csv+. Borrar \verb+hyg.tar.gz+.
	\part[5] Imprimir la entrada del catálogo correspondiente a la estrella \verb+Regulus+. Los nombres comunes de las estrellas están en la columna \verb+proper+. Debe usar \verb+grep+. 
	\part[10] ¿Cuántas estrellas registra el catálogo?
	\part[10] ¿Cuántas líneas, palabras y bytes tiene el catálogo?
	\part[15] ¿Cuántas estrellas son del tipo espectral (columna \verb+spect+) \textbf{G2}?
	\part[15] Crear un catálogo parcial con todas las estrellas de tipo G2 exclusivamente. Este catálogo debe tener el nombre \verb"EstrellasG2.csv" y debe incluir las primeras 32 líneas del catálogo completo.
	\part[15] Crear un catálogo parcial con todas las estrellas del mismo tipo espectral del Sol (G2V) que estén a  una distancia entre 10 y 50 años luz. Este catálogo debe tener el nombre \verb"EstrellasG2V_10-50_ly.csv" y no debe incluir las primeras 32 líneas del catálogo completo.
	\part[10] A cada una de las líneas del archivo \verb"EstrellasG2V_10-50_ly.csv" añada al comienzo de cada una la cadena de caracteres ``\verb+estrellas especiales,+'' y al final de ellas ``\verb+,enero-2015+''. Usar \verb+sed+.
	\part[5] Comprimir el catálogo parcial del punto anterior a \verb"EstrellasG2V_10-50_ly_ng.tar.gz".
\end{parts}

\end{questions}


\end{document}
