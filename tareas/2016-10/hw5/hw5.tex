
%--------------------------------------------------------------------
%--------------------------------------------------------------------
% Formato para los talleres del curso de Métodos Computacionales
% Universidad de los Andes
% 2015, curso de vacaciones
%--------------------------------------------------------------------
%--------------------------------------------------------------------

\documentclass[11pt,letterpaper]{exam}
\usepackage[utf8]{inputenc}
\usepackage[spanish]{babel}
\usepackage{graphicx}
\usepackage{enumerate}
\usepackage{tabularx}
\usepackage[absolute]{textpos} % Para poner una imagen en posiciones arbitrarias
\usepackage{multirow}
\usepackage{float}
\usepackage{hyperref}
%\usepackage{breakurl}
\decimalpoint


\begin{document}
\begin{center}

\includegraphics[width=16cm]{header.png}

\vspace{1.0cm}
{\Large Herramientas Computacionales - Tarea 5} \\
\textsc{Semana 6 - Funciones y recursividad}\\
2016-I\\
\end{center}

%\begin{textblock*}{40mm}(10mm,20mm)
%  \includegraphics[width=3cm]{logoUniandes.png}
%\end{textblock*}

%\begin{textblock*}{40mm}(164mm,20mm)
%  \includegraphics[width=3cm]{logoUniandes.png}
%\end{textblock*}

\vspace{0.5cm}

\noindent
El c\'odigo fuente debe subirse a Sicua plus como un \'unico archivo
\verb".py" con el nombre del estudiante en el formato \verb"NombreApellido_hw5.py" antes que termine la clase.

\vspace{0.5cm}

\begin{questions}
 
\question[5.0] {\bf{Tri\'angulo de Pascal}}

Escriba un script de Python que contenga una funci\'on que calcule los elementos de un tri\'angulo de Pascal recursivamente y los imprima hasta cierto orden n (dado como argumento a dicha funci\'on). Por ejemplo, si hago un llamado a \verb+ Pascal(4) +, se deber\'ia imprimir un mensaje en pantalla similar al siguiente:


\begin{center}

\begin{tabular}{rccccccccc}
$n=0$:&    &    &    &    &  1\\\noalign{\smallskip\smallskip}
$n=1$:&    &    &    &  1 &    &  1\\\noalign{\smallskip\smallskip}
$n=2$:&    &    &  1 &    &  2 &    &  1\\\noalign{\smallskip\smallskip}
$n=3$:&    &  1 &    &  3 &    &  3 &    &  1\\\noalign{\smallskip\smallskip}
$n=4$:&  1 &    &  4 &    &  6 &    &  4 &    &  1\\\noalign{\smallskip\smallskip}
\end{tabular}
\end{center}


Recuerde que debe definir el caso base y el caso recursivo correctamente. Tal vez la funci\'on necesite imprimir parte del tri\'angulo dentro de la misma y retornar la \'ultima fila como resultado de la funci\'on, eso est\'a permitido. 

\end{questions}

\end{document}
