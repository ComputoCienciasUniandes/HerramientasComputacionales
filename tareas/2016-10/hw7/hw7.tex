
%--------------------------------------------------------------------
%--------------------------------------------------------------------
% Formato para los talleres del curso de Métodos Computacionales
% Universidad de los Andes
% 2015, curso de vacaciones
%--------------------------------------------------------------------
%--------------------------------------------------------------------

\documentclass[11pt,letterpaper]{exam}
\usepackage[utf8]{inputenc}
\usepackage[spanish]{babel}
\usepackage{graphicx}
\usepackage{enumerate}
\usepackage{tabularx}
\usepackage[absolute]{textpos} % Para poner una imagen en posiciones arbitrarias
\usepackage{multirow}
\usepackage{float}
\usepackage{hyperref}
%\usepackage{breakurl}
\decimalpoint


\begin{document}
\begin{center}

\includegraphics[width=16cm]{header.png}

\vspace{1.0cm}
{\Large Herramientas Computacionales - Tarea 7} \\
\textsc{Semana 8 - Importaci\'on de m\'odulos y cuadernos de IPython}\\
2016-I\\
\end{center}

%\begin{textblock*}{40mm}(10mm,20mm)
%  \includegraphics[width=3cm]{logoUniandes.png}
%\end{textblock*}

%\begin{textblock*}{40mm}(164mm,20mm)
%  \includegraphics[width=3cm]{logoUniandes.png}
%\end{textblock*}

\vspace{0.5cm}

\noindent
La soluci\'on al problema debe subirse a Sicua Plus como una carpeta comprimida de extensi\'on \verb".zip" con el nombre del estudiante en el formato \verb"NombreApellido_hw7.zip" antes que termine la clase. Los archivos necesarios son el m\'odulo de Python y el cuaderno de IPython (extensi\'on \verb".ipynb")

\vspace{0.5cm}

\begin{questions}
 
\question[2.0] {\bf{Importar el m\'odulo}}

Crear un m\'etodo en el m\'odulo provisto con la tarea (\verb"baloncito.py") que de la soluc\'ion anal\'itica para el movimiento del bal\'on e importar el m\'odulo en el cuaderno de IPython.

\question[2.0] {\bf{Comportamiento en dt}}

Ver c\'omo se comporta el m\'etodo de Euler para 3 diferentes dt. Para eso graficar la diferencia entre la soluci\'on por el m\'etodo num\'erico y la soluci\'on anal\'itica en funci\'on del tiempo. 

\question[1.0] {\bf{The perks of IPython}}

Explicar lo que hizo y sus conclusiones usando las opciones de texto Markdown de IPython


\end{questions}

\end{document}
