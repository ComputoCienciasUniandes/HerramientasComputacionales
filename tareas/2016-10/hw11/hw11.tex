
%--------------------------------------------------------------------
%--------------------------------------------------------------------
% Formato para los talleres del curso de Métodos Computacionales
% Universidad de los Andes
% 2015, curso de vacaciones
%--------------------------------------------------------------------
%--------------------------------------------------------------------

\documentclass[11pt,letterpaper]{exam}
\usepackage[utf8]{inputenc}
\usepackage[spanish]{babel}
\usepackage{graphicx}
\usepackage{enumerate}
\usepackage{tabularx}
\usepackage[absolute]{textpos} % Para poner una imagen en posiciones arbitrarias
\usepackage{multirow}
\usepackage{float}
\usepackage{hyperref}
\usepackage{breakurl}
\decimalpoint

%\usepackage{pst-barcode}
%\usepackage{auto-pst-pdf}
\pointname{ puntos}

\begin{document}
\begin{center}

\includegraphics[width=16cm]{header.png}

\vspace{1.0cm}
{\Large Herramientas Computacionales \\
 Tarea 11} \\
\textsc{Semana 12 - Introducción a SciPy. Ajustes polinomiales y no polinomiales}\\
2016-I\\
\end{center}

%\begin{textblock*}{40mm}(10mm,20mm)
%  \includegraphics[width=3cm]{logoUniandes.png}
%\end{textblock*}

%\begin{textblock*}{40mm}(164mm,20mm)
%  \includegraphics[width=3cm]{logoUniandes.png}
%\end{textblock*}

\vspace{0.5cm}

{\Large Instrucciones de Entrega}\\

\noindent
La solución a este taller debe subirse por SICUA antes de terminar 
el horario de clase.
\noindent
Consiste de un IPython Notebook con el nombre
\verb"NombreApellido_hw11"
el cual debe contener todas las intrucciones necesarias del ejercicio.

\begin{questions}

\question[50]{\bf{Gaussiana + Recta}} 

Descargue el archivo de datos 
\href{https://raw.githubusercontent.com/ComputoCienciasUniandes/HerramientasComputacionalesDatos/master/Homework/hw11/gauss_line.txt}{gauss-line.txt}.
Haga una inspección visual del conjunto de datos y observe el comportamiento que tiene. Utilizando el módulo \verb|scipy.optimize.curve_fit| ajuste los datos a un modelo con cinco parámetros: una recta (dos parámetros) más una gaussiana (tres parámetros). Grafique el modelo de ajuste sobre los datos originales.

\question[50] {\bf{Temperaturas de Munich}} 

Descargue el archivo de temperaturas de Munich 
\href{http://www2.mpia-hd.mpg.de/~robitaille/PY4SCI_SS_2014/_static/data/munich_temperatures_average_with_bad_data.txt}{(munich.txt)} con el que ya se trabajó una vez. Empiece limpiando los datos (remueva los datos defectuosos, i.e. $T=99$ y $T =-99$).

Ajuste los datos a la función de tres parámetros:
$$ f(t; a,b,c) = a \cos\left( 2\pi t + b \right) + c $$

Grafique el modelo de ajuste sobre los datos originales ya limpios.

\end{questions}

\end{document}
