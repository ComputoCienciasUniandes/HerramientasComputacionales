
%--------------------------------------------------------------------
%--------------------------------------------------------------------
% Formato para los talleres del curso de Métodos Computacionales
% Universidad de los Andes
% 2015, curso de vacaciones
%--------------------------------------------------------------------
%--------------------------------------------------------------------

\documentclass[11pt,letterpaper]{exam}
\usepackage[utf8]{inputenc}
\usepackage[spanish]{babel}
\usepackage{graphicx}
\usepackage{enumerate}
\usepackage{tabularx}
\usepackage[absolute]{textpos} % Para poner una imagen en posiciones arbitrarias
\usepackage{multirow}
\usepackage{float}
\usepackage{hyperref}
%\usepackage{breakurl}
\decimalpoint


\begin{document}
\begin{center}

\includegraphics[width=16cm]{header.png}

\vspace{1.0cm}
{\Large Herramientas Computacionales - Tarea 4} \\
\textsc{Semana 5 - Caracteres, listas y estructuras iterativas}\\
2016-I\\
\end{center}

%\begin{textblock*}{40mm}(10mm,20mm)
%  \includegraphics[width=3cm]{logoUniandes.png}
%\end{textblock*}

%\begin{textblock*}{40mm}(164mm,20mm)
%  \includegraphics[width=3cm]{logoUniandes.png}
%\end{textblock*}

\vspace{0.5cm}

\noindent
El c\'odigo fuente debe subirse a Sicua plus como un \'unico archivo
\verb".py" con el nombre del estudiante en el formato \verb"NombreApellido_hw4.py" antes que termine la clase.

\vspace{0.5cm}

\begin{questions}
 
\question[1.0] {\bf{Descargar y leer}}

Escriba un script que descargue ``The Tragedy of Romeo and Juliet'' de William Shakespeare, cargue el texto en memoria y lo lea por l\'ineas, como se hace en el video de preparaci\'on.

\question[2.0]{\bf{Ciclo y separar}}

Haga un ciclo que recorra el texto l\'inea por l\'inea y las separe palabra por palabra con el fin de encontrar la palabra m\'as larga de todo el texto. 

\question[2.0]{\bf{Encontrar e imprimir}}

Encuentre correctamente la palabra m\'as larga de todo el texto e imprima cu\'al es.


\end{questions}

\end{document}
