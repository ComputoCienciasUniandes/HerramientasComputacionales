
%--------------------------------------------------------------------
%--------------------------------------------------------------------
% Formato para los talleres del curso de Métodos Computacionales
% Universidad de los Andes
% 2015-10
%--------------------------------------------------------------------
%--------------------------------------------------------------------

\documentclass[11pt,letterpaper]{exam}
\usepackage[utf8]{inputenc}
\usepackage[spanish]{babel}
\usepackage{graphicx}
\usepackage{enumerate}
\usepackage{tabularx}
\usepackage[absolute]{textpos} % Para poner una imagen completa en la portada
\usepackage{multirow}
\usepackage{float}
\usepackage{hyperref}
\usepackage{breakurl}

\decimalpoint
%\usepackage{pst-barcode}
%\usepackage{auto-pst-pdf}

\newcommand{\base}[1]{\underline{\hspace{#1}}}
\boxedpoints
\pointname{ pt}
%\extrawidth{0.75in}
%\extrafootheight{-0.5in}
\extraheadheight{-0.15in}
%\pagestyle{head}

%\noprintanswers
%\printanswers


\usepackage{upquote,textcomp}
\newcommand\upquote[1]{\textquotesingle#1\textquotesingle} % To fix straight quotes in verbatim

\begin{document}

\begin{center}

\vspace{1.0cm}
{\Large Herramientas Computacionales \\
 Tarea 12} \\
\textsc{Semana 13: Herramientas estadísticas} \\
Funciones estadísticas, histogramas y ejemplos de distribuciones\\
2016-I\\
\end{center}

\begin{textblock*}{40mm}(10mm,20mm)
  \includegraphics[width=3cm]{logoUniandes.png}
\end{textblock*}

\begin{textblock*}{40mm}(161mm,20mm)
  \includegraphics[width=3cm]{logoUniandes.png}
\end{textblock*}

\vspace{0.5cm}

{\Large Instrucciones de Entrega}\\

\noindent
La solución a este taller debe subirse por SICUA antes de terminar 
el horario de clase.
\noindent
Consiste de un IPython Notebook con el nombre
\verb"NombreApellido_hw12"
el cual debe contener todas las intrucciones necesarias del ejercicio.

\begin{questions}

\question[40]{\bf{Sumando números aleatorios}} 

Escriba una función que haga la suma de N números (cada uno entre 0 y 1) generados aleatoriamente (para eso puede usar la función \verb+random.random()+). A eso nos referimemos como un experimento y tendrá como resultado un número entre 0 y N. Ahora repita el experimento M veces guardando cada vez el resultado. Observe que se obtiene una distribución gaussiana. 

\question[60] {\bf{Ahora el experimento}} 

Ahora veamos cómo se comporta la distribución en función de N. Para eso fijaremos M=1000 y usamos tres valores diferentes para N, a saber 100, 1000 y 10000. Si su implementación se demora mucho en correr hágalo con 10, 100 y 1000. Ahora encuentre los parámetros para la distribución en cada caso haciendo un fit a una función gaussiana. Comente brevemente sobre los resultados observados.

\end{questions}

\end{document}

