
%--------------------------------------------------------------------
%--------------------------------------------------------------------
% Formato para los talleres del curso de Métodos Computacionales
% Universidad de los Andes
% 2015, curso de vacaciones
%--------------------------------------------------------------------
%--------------------------------------------------------------------

\documentclass[11pt,letterpaper]{exam}
\usepackage[utf8]{inputenc}
\usepackage[spanish]{babel}
\usepackage{graphicx}
\usepackage{tabularx}
\usepackage[absolute]{textpos} % Para poner una imagen en posiciones arbitrarias
\usepackage{multirow}
\usepackage{float}
\usepackage{hyperref}
\decimalpoint

\begin{document}
\begin{center}

\includegraphics[width=16cm]{header.png}

\vspace{1.0cm}
{\Large Herramientas Computacionales - Tarea 3} \\
\textsc{Semana 3 - Introducci\'on a Python}\\
2016-I\\
\end{center}

%\begin{textblock*}{40mm}(10mm,20mm)
%  \includegraphics[width=3cm]{logoUniandes.png}
%\end{textblock*}

%\begin{textblock*}{40mm}(164mm,20mm)
%  \includegraphics[width=3cm]{logoUniandes.png}
%\end{textblock*}

\vspace{0.5cm}

\noindent
El ejercicio ser\'a implementar un c\'odigo en Python que reciba un n\'umero del usuario a trav\'es de la terminal, luego determine si tal n\'umero es primo o no y finalmente imprima el resultado (si el n\'umero es o no primo). El c\'odigo fuente debe subirse a Sicua plus con el nombre del estudiante en el formato \verb"NombreApellido_hw3.py" antes que termine la clase. 

\vspace{0.5cm}

\begin{questions}
 
\question[1.0] {\bf{Captar el n\'umero}}

Escriba un script que reciba el n\'umero que quiera estudiar el usuario desde la terminal y lo guarde en una variable. Puede suponer que el n\'umero ser\'a entero.

\question[3.0]{\bf{Averiguar si es primo o no}}

En el mismo c\'odigo, implemente una soluci\'on para saber si el n\'umero es primo o no. Por facilidad puede hacer las pruebas con n\'umeros menores a 1000.

\question[1.0]{\bf{Imprima su soluci\'on}}

Imprima la soluci\'on, i.e., si el n\'umero es primo o no.

\end{questions}

\end{document}
