%Soluci\'on del taller 2
%Herramientas computacionales
%2018-1
%
%Soluci\'on preparada por David Guzm\'an

\documentclass[letterpaper]{article}

\begin{document}

\section{Informaci\'on del estudiante}

%Punto 2:

\begin{itemize}
  \item {\bf Nombre:} Juan P\'erez
  \item {\bf C\'odigo:} 201528945
  \item {\bf Programa:} arquitectura
  \item {\bf Edad:} 20 a\~nos
\end{itemize}

\section{Materias 2017-2}

%Punto 3:
\begin{table}[h!] %el comando h! indica que debe poner la tabla aqui (here)
 
\begin{tabular}{|l|r|r|}
 \hline
 Materia & Cr\'editos & Nota \\
 \hline
 C\'alculo diferencial & 3 & 3.5\\
 Constituci\'on y democracia & 3 & 5\\
 F\'isica 1 & 3 & 4.1\\
 F\'isica experimental 1 & 1 & 3.8\\
 Introducci\'on a la econom\'{\i}a colombiana & 3 & 4.3\\
 Qu\'{\i}mica general & 3 & 2.8\\
 Laboratorio de qu\'{\i}mica general & 1 & 4.5\\
 \hline
\end{tabular}
\caption[]{Notas 2017-2 de Juan P\'erez}\label{tab:notas}
\end{table}

\section{Promedio 2017-2}

%Punto 4a:

Para calcular el promedio ponderado del semestre se usa la ecuaci\'on

\begin{equation}
  \textrm{promedio} = \frac{\sum_{i=1}^m n_i \cdot c_i}{\sum_{i=1}^m c_i}
\end{equation}

donde $m$ es el n\'umero de materias inscritas, $n_i$ es la nota de cada materia y $c_i$ son los cr\'editos de cada materia.

%Punto 4b:
\begin{equation}
  \textrm{promedio} = \frac{(3.5\cdot3) + (5\cdot3) + (4.1\cdot3) + (3.8\cdot1) + (4.3\cdot3) + (2.8\cdot3) + (4.5\cdot1) }{3 + 3 + 3 + 1 + 3 + 3 + 1}
\end{equation}

\begin{equation}
  \textrm{promedio} = \frac{10.5 + 15 + 12.3 + 3.8 + 12.9 + 8.4 + 4.5}{17}
\end{equation}

\begin{equation}
  \textrm{promedio} = \frac{67.4}{17}
\end{equation}

\begin{equation}
  \textrm{promedio} = 3.96
\end{equation}

\end{document}