\documentclass[11pt,letterpaper]{exam}
\usepackage{amsmath}
\usepackage[utf8]{inputenc}
\usepackage[spanish]{babel} % español
\usepackage{graphicx}
\usepackage{tabularx}
\usepackage[absolute]{textpos} % Para poner una imagen en posiciones arbitrarias
\usepackage{multirow}
\usepackage{float}
\usepackage{hyperref}
\usepackage{breakurl}
\usepackage[spanish]{babel}

\decimalpoint	%para que los n�meros decimales est�n separados por punto


\pointpoints{punto}{puntos}	%para que los textos de puntajes salgan en español. http://www-math.mit.edu/~psh/exam/examdoc.pdf
\footer{}{P\'agina \thepage}{}%para que pie de pagina salga en español


\begin{document}

\begin{center}

\includegraphics[width=16cm]{header.png}

\vspace{1.0cm}
{\Large Herramientas Computacionales - Taller 11} \\
\textsc{Semana 13 - Herramientas estad\'isticas}\\
2018-I\\
\end{center}


\vspace{0.5cm}

\noindent

La soluci\'on debe subirse a SicuaPlus en un \'unico archivo de \verb"IPython Notebook" con el nombre
\verb"NombreApellido_taller11.ipynb", el cual debe contener toda la soluci\'on del taller.\\ 

\vspace{0.3cm}

\begin{questions}

\question[0.5] {\bf{Comentarios}}

Comente su c\'odigo.

\LARGE \textbf{El Teorema del L\'imite Central}\\

\normalsize

En un juego de mesa hay un momento crucial de la partida en que un jugador debe lanzar 10 veces un dado de 6 caras y sumar todos sus resultados. El dado est\'a cargado de tal manera que la probabilidad de obtener cada n\'umero es $P([1,2,3,4,5,6]) = [0.2,0.1,0.2,0.3,0.1,0.1]$. El jugador gana la partida si obtiene una suma mayor a 40.\\

En este taller queremos verificar cualitativamente cu\'anto es la probabilidad de obtener una suma mayor a 40. Para ello tenemos que simular muchas ($M\approx 10000$) sumas y realizar un histograma. La raz\'on entre el n\'umero de sumas que caen en una barra y el n\'umero total es aproximadamente la probabilidad de obtener una suma en el intervalo de dicha barra.\\

\question[2.0] {\bf{Generar datos}}

Para simular muchas sumas de lanzamiento de dados, primero tenemos que simular los lanzamientos del dado cargado. Para esto, cree una funci\'on llamada \verb"get_sum(N)" que dado un n\'umero entero $N$ retorne la suma de $N$ n\'umeros aleatorios siguiendo la distribuci\'on $P([1,2,3,4,5,6]) = [0.2,0.1,0.2,0.3,0.1,0.1]$.\\

\textbf{Ayuda:} Revise la documentaci\'on de numpy.random.choice.\\

\question[1.5] {\bf{Generar datos 2}}

Ahora que podemos simular la suma de $N=10$ lanzamientos de dados, debemos repetir el procedimiento para simular $M=10000$ sumas y de esta manera poder realizar el histograma.\\

Para esto, cree una funci\'on llamada \verb"repeat_sum(M,N)" que llame $M$ veces la funci\'on \verb"get_sum(N)", cada vez guardando el resultado en una lista. Esta funcion debe retornar una lista de $M$ elementos donde cada elemento es el resultado de una suma $S_N$, dada por la funci\'on \verb"get_sum(N)".\\

\question[1.0] {\bf{An\'alisis de datos y gr\'aficas}}

Una de las herramientas m\'as poderosas de la estad\'istica es el Teorema del L\'imite Central (TLC). Este teorema establece que si generamos un n\'umero aleatorio $N$ veces y sumamos estos valores, la probabilidad de obtener cierto valor de la suma est\'a aproximado por una gaussiana sin importar la distribuci\'on de probabilidad de donde vengan los n\'umeros.\\

En este caso, el TLC nos dice que sin importar c\'omo el dado est\'e cargado, la probabilidad de obtener una suma $S_N$ de $N$ lanzamientos de dados, est\'a dada por una gaussiana:

\begin{equation}
P(S_N) \approx  \sqrt{\frac{1}{2\pi\sigma^2}}\exp\left[{-\frac{\left(S_N-\bar{\chi}\right)^2}{\sigma^2}}\right],
\end{equation}

donde $\sigma$ es la desviaci\'on est\'andar y $\bar{\chi}$ es el promedio.\\

Usando la funci\'on del punto $3$, obtenga $M=10000$ valores $S_N$ con $N=10$. Imprima un mensaje que deje claro cu\'al el valor promedio y la desviaci\'on est\'andar de la lista de 10000 valores $S_N$.\\

Grafique el histograma normalizado de esta lista de valores. Adicionalmente grafique la curva gaussiana con los valores que acaba de obtener.\\

As\'i podemos verificar las posibilidades que tiene este jugador de ganar la partida.\\

\textbf{Ayuda:} Utilice la funci\'on \verb"plt.hist(lista_vals, bins = 20, normed = True)" para graficar un histograma normalizado de 20 barras a partir de la lista llamada \verb"lista_vals"

\textbf{Ayuda:} Grafique primero el histograma y luego verifique el rango de valores para poder graficar la gaussiana haciendo uso de un \verb"linspace".

\end{questions}

\end{document}
