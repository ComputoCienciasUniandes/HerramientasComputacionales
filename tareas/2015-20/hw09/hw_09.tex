
%--------------------------------------------------------------------
%--------------------------------------------------------------------
% Formato para los talleres del curso de Métodos Computacionales
% Universidad de los Andes
% 2015-10
%--------------------------------------------------------------------
%--------------------------------------------------------------------

\documentclass[11pt,letterpaper]{exam}
\usepackage[utf8]{inputenc}
\usepackage[spanish]{babel}
\usepackage{graphicx}
\usepackage{enumerate}
\usepackage{tabularx}
\usepackage[absolute]{textpos} % Para poner una imagen completa en la portada
\usepackage{multirow}
\usepackage{float}
\usepackage{hyperref}
\usepackage{breakurl}

\decimalpoint
%\usepackage{pst-barcode}
%\usepackage{auto-pst-pdf}

\newcommand{\base}[1]{\underline{\hspace{#1}}}
\boxedpoints
\pointname{ pt}
%\extrawidth{0.75in}
%\extrafootheight{-0.5in}
\extraheadheight{-0.15in}
%\pagestyle{head}

%\noprintanswers
%\printanswers


\usepackage{upquote,textcomp}
\newcommand\upquote[1]{\textquotesingle#1\textquotesingle} % To fix straight quotes in verbatim

\begin{document}

\begin{center}
{\Large Herramientas Computacionales \\
 Taller 9}\\
Profesores: \\ Felipe G\'omez\\ Juan David Orjuela \\
Fecha de Publicación: {\small \it Octubre 6 de 2015}\\
\end{center}

\begin{textblock*}{40mm}(10mm,20mm)
  \includegraphics[width=3cm]{logoUniandes.png}
\end{textblock*}

\begin{textblock*}{40mm}(161mm,20mm)
  \includegraphics[width=3cm]{logoUniandes.png}
\end{textblock*}

\vspace{0.5cm}

{\Large Instrucciones de Entrega}\\

\noindent
La solución a este taller debe subirse por SICUA antes de terminar 
el horario de clase.
\noindent
Primero debe crearse una carpeta de trabajo llamada \verb"NombreApellido_hw9"
dentro de la cual deben estar los siguientes archivos: un Ipython notebook 
y una gráfica en formato png. Una vez haya terminado de trabajar, debe 
comprimir la carpeta desde la consola con el comando:

\verb"zip -r NombreApellido_hw9.zip NombreApellido_hw9"

\noindent Debe enviar el archivo comprimido \verb"NombreApellido_hw9.zip" por SICUA. 
Es importante realizar estos pasos correctamente, ya que se calificará con un
script que asigna la nota 0.0 si los archivos no están correctamente nombrados.

\noindent La imágen \verb"circulos.png" se encuentra disponible en el siguente \burlalt{enlace.}{https://raw.githubusercontent.com/ComputoCienciasUniandes/HerramientasComputacionalesDatos/master/Homework/hw09/circulos.png}

\begin{questions}

\question[70] {\bf{Encontrar centros de masa}} 

Cree un Ipython Notebook llamado \verb"centros_de_masa.ipynb", allí
debe leer el archivo de imágen \verb"circulos.png" que tiene tres círculos blancos 
sobre un fondo en negro como un array. Muéstrela con el método
\verb"imshow()"

Siguiendo el concepto de centro de masa, calcule el centro de cada círculo.

Imprima los resultados en una celda utilizando pixeles como unidad de medidad.


\question[30]{\bf{Gráfica}} 

Utilizando los comandos \verb"axvline()" y \verb"axhline()" trace líneas verticales
y horizontales que muestren los centros de los tres círculos. 
Grabe la imágen que usted ha creado con \texttt{savefig("solucion.png")}

\begin{figure}[hb]
\begin{centering}
\includegraphics[scale=0.5]{sol}
\caption{Ejemplo de marcado del centro de un círculo.}
\end{centering}
\end{figure}
\end{questions}


\end{document}
