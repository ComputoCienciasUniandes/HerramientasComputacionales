
%--------------------------------------------------------------------
%--------------------------------------------------------------------
% Formato para los talleres del curso de Métodos Computacionales
% Universidad de los Andes
% 2015-10
%--------------------------------------------------------------------
%--------------------------------------------------------------------

\documentclass[11pt,letterpaper]{exam}
\usepackage[utf8]{inputenc}
\usepackage[spanish]{babel}
\usepackage{graphicx}
\usepackage{enumerate}
\usepackage{tabularx}
\usepackage[absolute]{textpos} % Para poner una imagen completa en la portada
\usepackage{multirow}
\usepackage{float}
\usepackage{hyperref}
\usepackage{breakurl}

\decimalpoint
%\usepackage{pst-barcode}
%\usepackage{auto-pst-pdf}

\newcommand{\base}[1]{\underline{\hspace{#1}}}
\boxedpoints
\pointname{ pt}
%\extrawidth{0.75in}
%\extrafootheight{-0.5in}
\extraheadheight{-0.15in}
%\pagestyle{head}

%\noprintanswers
%\printanswers


\usepackage{upquote,textcomp}
\newcommand\upquote[1]{\textquotesingle#1\textquotesingle} % To fix straight quotes in verbatim

\begin{document}

\begin{center}
{\Large Herramientas Computacionales \\
 Taller 12}\\
Profesores: \\ Felipe G\'omez\\ Juan David Orjuela \\
Fecha de Publicación: {\small \it Octubre 27 de 2015}\\
\end{center}

\begin{textblock*}{40mm}(10mm,20mm)
  \includegraphics[width=3cm]{logoUniandes.png}
\end{textblock*}

\begin{textblock*}{40mm}(161mm,20mm)
  \includegraphics[width=3cm]{logoUniandes.png}
\end{textblock*}

\vspace{0.5cm}

{\Large Instrucciones de Entrega}\\

\noindent
La solución a este taller debe subirse por SICUA antes de terminar 
el horario de clase.
\noindent
Consiste de un IPython Notebook con el nombre
\verb"NombreApellido_hw12"
el cual debe contener todas las intrucciones necesarias del ejercicio.

Es importante realizar estos pasos correctamente, ya que se calificará con un
script que asigna la nota 0.0 si los archivos no están correctamente nombrados.

\begin{questions}

\question[100]{\bf{Gaussiana + Recta}} 

Descargue el archivo de datos 
\href{https://github.com/ComputoCienciasUniandes/HerramientasComputacionales/raw/master/Homeworks/2015-20/hw12/gauss_line.txt}{gauss-line.txt}.
Utilizando el módulo \verb|scipy.optimize.curve_fit| ajuste los datos a un modelo que 
tenga una recta (dos parámetros) y una gaussiana (tres parámetros).

Grafique los datos originales y el modelo de ajuste.



\question[50] {\bf{BONO - Munich}} 

Descargue el archivo de temperaturas de Munich 
\href{http://www2.mpia-hd.mpg.de/~robitaille/PY4SCI_SS_2014/_static/data/munich_temperatures_average_with_bad_data.txt}{(munich.txt)} y
remueva los datos defectuosos ($T=99$ y $T =-99$).

Ajuste los datos a la función de tres parámetros:
$$ f(t; a,b,c) = a \cos\left( 2\pi t + b \right) + c $$

Grafique los datos originales (limpios) y el modelo de ajuste.


\end{questions}

\end{document}

