\documentclass[letterpaper,10pt,onecolumn]{article}
\usepackage[spanish]{babel}
\usepackage[utf8]{inputenc}
\usepackage{amsfonts}
\usepackage{amsthm}
\usepackage{amsmath}
\usepackage{mathrsfs}
%\usepackage{empheq}
\usepackage{enumitem}
\usepackage[pdftex]{color,graphicx}
\usepackage{hyperref}
\usepackage{listings}
\usepackage{calligra}
%\usepackage{algpseudocode} 
\DeclareMathAlphabet{\mathcalligra}{T1}{calligra}{m}{n}
\DeclareFontShape{T1}{calligra}{m}{n}{<->s*[2.2]callig15}{}
\newcommand{\scripty}[1]{\ensuremath{\mathcalligra{#1}}}
\lstloadlanguages{[5.2]Mathematica}
\setlength{\oddsidemargin}{0cm}
\setlength{\textwidth}{490pt}
\setlength{\topmargin}{-40pt}
\addtolength{\hoffset}{-0.3cm}
\addtolength{\textheight}{4cm}

%%%%%%%%%%%%%%%%%%%%%%%%%

\usepackage{marvosym} %use this pack
\usepackage{xcolor} %for color
\usepackage{dtklogos}

%%%%%%%%%%%%%%%%%%%%%%%%%

\begin{document}
\begin{center}

\includegraphics[width=490pt]{header.png}\\[0.5cm]

\textsc{\huge Herramientas Computacionales}\\[0.1cm]

\Large \textsc{Contenido Programático}\\[0.7cm]

\end{center}

\large \noindent\textsc{Nombre del curso:} Herramientas Computacionales
	 
\noindent\textsc{Código del curso:} FISI 2026

\noindent\textsc{Unidad académica:} Departamento de Física

\noindent\textsc{Prerrequisitos:} Algorítmica y Programación Orientada por Objetos 1 (ISIS 1204)

\noindent\rule{\textwidth}{1pt}\\[-0.1cm]

\newcounter{mysection}
\addtocounter{mysection}{1}

\noindent\textbf{\large \Roman{mysection} \quad Introducción}\\[-0.2cm]

\noindent\normalsize Los computadores nos ayudan a organizar,
comunicar y procesar información, y hoy en día son esenciales en los
mundos de la ciencia, la academia y la técnica. Este curso enseña
algunas herramientas computacionales básicas para hacer de los
computadores herramientas útiles, poderosas y versátiles. El curso
desarrolla habilidades de programación en un lenguaje de alto nivel,
por ejemplo Python o Matlab; enseña algunos métodos de análisis
numérico; y exhibe algunas herramientas útiles en el análisis de
datos. \\[0.1cm] 

\stepcounter{mysection}
\noindent\textbf{\large \Roman{mysection} \quad Objetivos}\\[-0.2cm]

\noindent\normalsize Los objetivos del curso son:

\begin{itemize}
	\item Ofrecer herramientas computacionales básicas útiles en la investigación y la vida académica.\\[-0.6cm]
	\item Introducir rutinas sencillas de análisis numérico.\\[-0.6cm]
	\item Desarrollar habilidades de programación.\\[-0.2cm]
\end{itemize}

\stepcounter{mysection}
\noindent\textbf{\large \Roman{mysection} \quad Competencias}\\[-0.2cm]

\noindent\normalsize Al finalizar el curso, se espera que el estudiante esté en capacidad de:

\begin{itemize}
	\item Utilizar computadores con sistema operativo tipo UNIX.\\[-0.6cm]
	\item Preparar documentos usando el sistema de composición de textos \LaTeX.\\[-0.6cm]
	\item Implementar en un lenguaje de programación de alto nivel la solución de problemas computacionales sencillos.\\[-0.6cm]
	\item Manipular, analizar y visualizar datos usando un lenguaje de programación de alto nivel.\\[-0.2cm]
\end{itemize}

\stepcounter{mysection}
\noindent\textbf{\large \Roman{mysection} \quad Contenido}\\[-0.2cm]

\noindent\normalsize \textbf{\textsc{Semanas 1}} [Linux]
Introducción a UNIX: filosofía, comandos básicos.
\\[-0.3cm] 

\noindent\textbf{\textsc{Semanas 2}}  [Linux] Editores de texto,
control de procesos, redirección. \\[-0.3cm]  

\noindent\textbf{\textsc{Semanas 3}} [\LaTeX] Lógica de
compilación, tipos de documentos.

\noindent\textbf{\textsc{Semanas 4}} [\LaTeX] Secciones, ecuaciones, tablas y
figuras. Manejo de bibliografía con \BibTeX.\\[-0.3cm]   

\noindent\textbf{\textsc{Semana 5}} [Python] Introducción: filosofía,
sintaxis básica, operaciones aritméticas, condicionales.\\[-0.3cm]   

\noindent\textbf{\textsc{Semana 6}} [Python] Operaciones con cadenas de
caracteres, listas y estructuras iterativas.\\[-0.3cm] 

\noindent\textbf{\textsc{Semana 7}} [Python] Definición de funciones,
tipos de variables, recursividad. \\[-0.3cm] 

\noindent\textbf{\textsc{Semana 8}} [Python] Programaci\'on orientada
a objetos en Python.\\[-0.3cm]  

\noindent\textbf{\textsc{Semana 9}} [Python] Instalación (\verb+pip+) e
importación de módulos. Cuadernos de iPython.\\[-0.3cm]  

\noindent\textbf{\textsc{Semanas 10}} [Python] Introducción a
NumPy. Arrays de numpy y operaciones entre arrays. \\[-0.3cm]  

\noindent\textbf{\textsc{Semanas 11}} [Python] Importación de datos. Visualización de datos con
\verb+matplotlib+: \verb+plot, scatter, imshow, subplot+.\\[-0.3cm]  

\noindent\textbf{\textsc{Semanas 12}} [Python] Análisis numérico:
métodos de bisección, método de Newton-Raphson. \\[-0.3cm]  

\noindent\textbf{\textsc{Semanas 13}} [Python] Introducción a
SciPy. Ajustes polinomiales y no polinomiales.\\[-0.3cm]  

\noindent\textbf{\textsc{Semana 14}} [Python] Herramientas
estadísticas: funciones estadísticas, histogramas y ejemplos de
distribuciones.\\[-0.3cm] 

\noindent\textbf{\textsc{Semana 15}} [Python] Métodos de Monte Carlo:
integración.\\[-0.3cm] 

\stepcounter{mysection}
\noindent\textbf{\large \Roman{mysection} \quad Metodología}\\[-0.2cm]

\noindent\normalsize Se har\'a \'efasis en el trabajo individual por
fuera del horarior de clase. El profesor entregar\'a una lista de
recursos que cada estudiante debe preparar. 
En la primera parte de cada clase el profesor
hace una sesi\'on demostrativa y en la segunda los
estudiantes comienzan a resolver los ejercicios de la semana. 
Todo el trabajo es individual y las calificaciones se otorgar\'an por
los resultados entregados exclusivamente en horario de clase.
\\[0.1cm] 



\stepcounter{mysection}
\noindent\textbf{\large \Roman{mysection} \quad Calificaci\'on del curso}\\[-0.2cm]

\begin{itemize}
	\item Entregas en clase $100\%$
\end{itemize}
\noindent\normalsize En 13 clases durante el semestre se deben
entregar ejercicios. 
Estos ejercicios se deben resolver y entregar antes de la
finalizaci\'on de la clase por Sicuaplus.
Al final se quitar\'a la peor y la mejor nota de los talleres y los
restantes ser\'an los que se promediar\'an. 
\\[0.1cm] 

\stepcounter{mysection}
\noindent\textbf{\large \Roman{mysection} \quad Bibliografía}\\[-0.2cm]

\noindent\normalsize Bibliografía principal:

%\href{http://www.wikibooks.org}{\color{blue}\looparrowright}

\begin{itemize}
	\item H. P. Langtangen. \textit{A Primer on Scientific Programming with Python}, 2009.\\
	\href{http://link.springer.com.ezproxy.uniandes.edu.co:8080/book/10.1007\%2F978-3-642-18366-9}{\nolinkurl{http://link.springer.com.ezproxy.uniandes.edu.co:8080/book/10.1007\%2F978-3-642-18366-9}}\\[-0.6cm]
\end{itemize}

\noindent\normalsize Bibliografía complementaria:

\begin{itemize}
	\item J. V. Guttag. \textit{Introduction to Computation and Programming Using Python}, 2013.
	\item K. D. Lee. \textit{Python Programming Fundamentals}, 2011. \\
	\href{http://link.springer.com.ezproxy.uniandes.edu.co:8080/book/10.1007\%2F978-1-84996-537-8}{\nolinkurl{http://link.springer.com.ezproxy.uniandes.edu.co:8080/book/10.1007\%2F978-1-84996-537-8}}\\[-0.6cm]
	\item S. van Vugt. \textit{Beginning the Linux Command Line}, 2009.\\
	\href{http://link.springer.com.ezproxy.uniandes.edu.co:8080/book/10.1007\%2F978-1-4302-1890-6}{\nolinkurl{http://link.springer.com.ezproxy.uniandes.edu.co:8080/book/10.1007\%2F978-1-4302-1890-6}}\\[-0.6cm]
	\item G. Gr\"atzer. \textit{More Math Into  \LaTeX}, 2007.\\
	\href{http://link.springer.com.ezproxy.uniandes.edu.co:8080/book/10.1007\%2F978-0-387-68852-7}{\nolinkurl{http://link.springer.com.ezproxy.uniandes.edu.co:8080/book/10.1007\%2F978-0-387-68852-7}}
\end{itemize}

\end{document}
