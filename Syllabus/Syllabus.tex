\documentclass[letterpaper,10pt,onecolumn]{article}
\usepackage[spanish]{babel}
\usepackage[utf8]{inputenc}
\usepackage{amsfonts}
\usepackage{amsthm}
\usepackage{amsmath}
\usepackage{mathrsfs}
%\usepackage{empheq}
\usepackage{enumitem}
\usepackage[pdftex]{color,graphicx}
\usepackage{hyperref}
\usepackage{listings}
\usepackage{calligra}
%\usepackage{algpseudocode} 
\DeclareMathAlphabet{\mathcalligra}{T1}{calligra}{m}{n}
\DeclareFontShape{T1}{calligra}{m}{n}{<->s*[2.2]callig15}{}
\newcommand{\scripty}[1]{\ensuremath{\mathcalligra{#1}}}
\lstloadlanguages{[5.2]Mathematica}
\setlength{\oddsidemargin}{0cm}
\setlength{\textwidth}{490pt}
\setlength{\topmargin}{-40pt}
\addtolength{\hoffset}{-0.3cm}
\addtolength{\textheight}{4cm}

%%%%%%%%%%%%%%%%%%%%%%%%%

\usepackage{marvosym} %use this pack
\usepackage{xcolor} %for color
%\usepackage{dtklogos}

%%%%%%%%%%%%%%%%%%%%%%%%%

\begin{document}
\begin{center}

\includegraphics[width=490pt]{header.png}\\[0.5cm]

\textsc{\huge Herramientas Computacionales}\\[0.1cm]

\Large \textsc{Contenido Programático}\\[0.7cm]

\end{center}

\large \noindent\textsc{Nombre del curso:} Herramientas Computacionales
	 
\noindent\textsc{Código del curso:} FISI 2026

\noindent\textsc{Unidad académica:} Departamento de Física

\noindent\textsc{Prerrequisitos:} Algorítmica y Programación Orientada por Objetos 1 (ISIS 1204)

\noindent\rule{\textwidth}{1pt}\\[-0.1cm]

\newcounter{mysection}
\addtocounter{mysection}{1}

\noindent\textbf{\large \Roman{mysection} \quad Introducción}\\[-0.2cm]

\noindent\normalsize Los computadores nos ayudan a organizar,
comunicar y procesar información, y hoy en día son esenciales en los
mundos de la ciencia, la academia y la técnica. Este curso enseña
algunas herramientas computacionales básicas para hacer de los
computadores herramientas útiles, poderosas y versátiles. El curso
desarrolla habilidades de programación en un lenguaje de alto nivel,
por ejemplo Python o Matlab; enseña algunos métodos de análisis
numérico; y exhibe algunas herramientas útiles en el análisis de
datos. \\[0.1cm] 

\stepcounter{mysection}
\noindent\textbf{\large \Roman{mysection} \quad Objetivos}\\[-0.2cm]

\noindent\normalsize Los objetivos del curso son:

\begin{itemize}
	\item Ofrecer herramientas computacionales básicas útiles en la investigación y la vida académica.\\[-0.6cm]
	\item Introducir rutinas sencillas de análisis numérico.\\[-0.6cm]
	\item Desarrollar habilidades de programación.\\[-0.2cm]
\end{itemize}

\stepcounter{mysection}
\noindent\textbf{\large \Roman{mysection} \quad Competencias}\\[-0.2cm]

\noindent\normalsize Al finalizar el curso, se espera que el estudiante esté en capacidad de:

\begin{itemize}
	\item Utilizar computadores con sistema operativo tipo UNIX.\\[-0.6cm]
	\item Preparar documentos usando el sistema de composición de textos \LaTeX.\\[-0.6cm]
	\item Implementar en un lenguaje de programación de alto nivel la solución de problemas computacionales sencillos.\\[-0.6cm]
	\item Manipular, analizar y visualizar datos usando un lenguaje de programación de alto nivel.\\[-0.2cm]
\end{itemize}

\stepcounter{mysection}
\noindent\textbf{\large \Roman{mysection} \quad Contenido}\\[-0.2cm]

Se muestran los temas y los videos que deben ser vistos antes de la clase correspondiente a la semana. El enlace a la lista de reproducci\'on es:

\url{https://www.youtube.com/playlist?list=PLHQtzvthdVM_MGC9dPFKe4hPAwBd_7RJ3}

\noindent\normalsize \textbf{\textsc{Semana 1}} [Introducción]
Presentación y examen diagnóstico.
\\[-0.3cm] 

\noindent\textbf{\textsc{Semana 2}} [Linux]
Introducción a UNIX: filosofía, comandos básicos.
\\[-0.3cm] 

\noindent\textbf{\textsc{Semana 3}}  [Linux] Editores de texto, control de procesos, redirección.

\textbf{Videos:} 1 y 2 Introducci\'on a UNIX (Ubuntu).
\\[-0.3cm]  

\noindent\textbf{\textsc{Semana 4}} [\LaTeX] Lógica de
compilación, tipos de documentos, secciones, ecuaciones, tablas y
figuras. Manejo de bibliografía con BibTeX.

\textbf{Video:} 3 Introducci\'on a \LaTeX.
\\[-0.3cm]   

\noindent\textbf{\textsc{Semana 5}} [Python] Introducción: filosofía,
sintaxis básica, operaciones aritméticas, condicionales.

\textbf{Videos:} 4 y 5 Introducci\'on a Python.
\\[-0.3cm]   

\noindent\textbf{\textsc{Semana 6}} [Python] Operaciones con cadenas de
caracteres, listas y estructuras iterativas.

\textbf{Video:} 6  Python: Listas y strings.
\\[-0.3cm] 

\noindent\textbf{\textsc{Semana 7}} [Python] Definición de funciones,
tipos de variables, recursividad. 

\textbf{Video:} 7 Python: Funciones, tipos de variables y recursividad.
\\[-0.3cm] 

\noindent\textbf{\textsc{Semana 8}} [Python] Programaci\'on orientada a objetos en Python. Instalación (\verb+pip+) e importación de módulos. Cuadernos de iPython.

\textbf{Videos:} 8 Programación Orientada a Objetos en Python y 9 Python: Módulos, PIP y cuadernos de Jupyter.
\\[-0.3cm]  

\noindent\textbf{\textsc{Semana 9}} [Python] Introducción a NumPy. Arrays de numpy y operaciones entre arrays. 

\textbf{Video:} 10 Numpy con IPython
\\[-0.3cm]  

\noindent\textbf{\textsc{Semana 10}} [Python] Importación de datos. Visualización de datos con
\verb+matplotlib+: \verb+plot, scatter, imshow+ y  \verb+ subplot+.

\textbf{Video:} 11 import Matplotlib as plt.
\\[-0.3cm]  

\noindent\textbf{\textsc{Semana 11}} [Python] Análisis numérico: métodos de bisección, método de Newton-Raphson. 

\textbf{Video:} 12 Métodos numéricos: ceros de funciones con IPython.
\\[-0.3cm]  

\noindent\textbf{\textsc{Semana 12}} [Python] Introducción a
SciPy. Ajustes polinomiales y no polinomiales.

\textbf{Video:} 13 Ajuste a funciones lineales y no lineales.
\\[-0.3cm]  

\noindent\textbf{\textsc{Semana 13}} [Python] Herramientas
estadísticas: funciones estadísticas, histogramas y ejemplos de
distribuciones.

\textbf{Video:} 14 Herramientas estadísticas con Python.
\\[-0.3cm] 

\noindent\textbf{\textsc{Semana 14}} [Python] Métodos de Monte Carlo:
integración.

\textbf{Video:} 15 M\'etodos Monte Carlo y 16 Integraci\'on Monte Carlo.
\\[-0.3cm] 

\noindent\textbf{\textsc{Semana 15}} [Python] Examen final. \\[0.1cm] 

\stepcounter{mysection}
\noindent\textbf{\large \Roman{mysection} \quad Metodología}\\[-0.2cm]

\noindent\normalsize Se har\'a \'enfasis en el trabajo individual por
fuera del horarior de clase. El profesor entregar\'a una lista de
recursos que cada estudiante debe preparar. 
Al inicio de cada clase el profesor responderá preguntas y 
hará aclaraciones sobre el material de la semana. Luego los
estudiantes resolverán los ejercicios de la semana individualmente. 
Todo el trabajo es individual y las calificaciones se otorgar\'an por
los resultados entregados exclusivamente en horario de clase por SICUA. 
Bajo ninguna circunstancia se recibirán ejercicios entregados por fuera 
del tiempo estipulado (hora oficial del final de la clase más 10 minutos).
Se recomienda traer a clase una memoria USB si le interesa guardar su 
trabajo. Se espera del estudiante que evite activamente que
su trabajo sea plagiado: cualquier sospecha de copia hará que el caso sea
llevado automáticamente a proceso disciplinario, por lo que se aconseja
que borre sus archivos del computador que usó cuando no los necesite más.
La memoria USB también podrá ser útil si por algún motivo su computador
pierde la conexión a internet y necesita cambiar de ordenador para subir
su trabajo a la plataforma de SICUA.
\\[0.1cm] 


\stepcounter{mysection}
\noindent\textbf{\large \Roman{mysection} \quad Calificaci\'on del curso}\\[-0.2cm]

\begin{itemize}
	\item Examen diagnóstico $20\%$
	\item 3 entregas en clase $60\%$
	\item Examen final $20\%$
\end{itemize}
\noindent\normalsize En doce clases durante el semestre se deberá entregar 
ejercicios. Estos ejercicios se deben resolver y subir antes de la
finalizaci\'on de la clase a Sicuaplus. Entre esos doce ejercicios se 
eligirán tres al azar que serán calificados, cada uno con un peso del $20\%$
de la nota final. Con el simple hecho de entregar el examen diagnóstico
tendrá la nota máxima en ese ítem.
\\[0.1cm] 

\stepcounter{mysection}
\noindent\textbf{\large \Roman{mysection} \quad Bibliografía}\\[-0.2cm]

\noindent\normalsize Bibliografía principal:

%\href{http://www.wikibooks.org}{\color{blue}\looparrowright}

\begin{itemize}
	\item H. P. Langtangen. \textit{A Primer on Scientific Programming with Python}, 2009.\\
	\href{http://link.springer.com.ezproxy.uniandes.edu.co:8080/book/10.1007\%2F978-3-642-18366-9}{\nolinkurl{http://link.springer.com.ezproxy.uniandes.edu.co:8080/book/10.1007\%2F978-3-642-18366-9}}\\[-0.6cm]
\end{itemize}

\noindent\normalsize Bibliografía complementaria:

\begin{itemize}
	\item J. V. Guttag. \textit{Introduction to Computation and Programming Using Python}, 2013.
	\item K. D. Lee. \textit{Python Programming Fundamentals}, 2011. \\
	\href{http://link.springer.com.ezproxy.uniandes.edu.co:8080/book/10.1007\%2F978-1-84996-537-8}{\nolinkurl{http://link.springer.com.ezproxy.uniandes.edu.co:8080/book/10.1007\%2F978-1-84996-537-8}}\\[-0.6cm]
	\item S. van Vugt. \textit{Beginning the Linux Command Line}, 2009.\\
	\href{http://link.springer.com.ezproxy.uniandes.edu.co:8080/book/10.1007\%2F978-1-4302-1890-6}{\nolinkurl{http://link.springer.com.ezproxy.uniandes.edu.co:8080/book/10.1007\%2F978-1-4302-1890-6}}\\[-0.6cm]
	\item G. Gr\"atzer. \textit{More Math Into  \LaTeX}, 2007.\\
	\href{http://link.springer.com.ezproxy.uniandes.edu.co:8080/book/10.1007\%2F978-0-387-68852-7}{\nolinkurl{http://link.springer.com.ezproxy.uniandes.edu.co:8080/book/10.1007\%2F978-0-387-68852-7}}
\end{itemize}

\end{document}
