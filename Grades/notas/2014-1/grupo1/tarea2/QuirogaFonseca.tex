
\documentclass[english]{article}
\usepackage[T1]{fontenc}
\usepackage[utf8]{luainputenc}
\usepackage{amstext}
\usepackage{babel}
\begin{document}

\title{Notas de Clase}


\author{Valentina Quiroga Fonseca}

\maketitle

\section{Fisica Estadistica}

Calculo de estados accesibles a partir de probabilidades continuas. 

Calcularlo a partir de una distribución continua para caso tridimensional:

Sea O(A) el número de estados accesibles para un sistema dado. Luego:

O= BV$^{N}$A$^{\frac{3N}{2}}$

Con una probabilidad correspondiente a:

P(y)=$\frac{BV^{N}A^{\frac{3N}{2}}}{B(2V)^{N}A^{\frac{3N}{2}}}$=$\frac{1}{2^{N}}$
donde N corresponde al Numero de Avogadro.

Para interacciones caloricas, la ocupación de estados cambia, luego:

\[
\frac{\partial F}{\partial y\partial x}\neq\frac{\partial F}{\partial x\partial y}
\]


Es decir, corresponde a un diferencial inexacto. 

Para un sistema aislado, la probabilidad condicionada dada una cierta
energia es:

P(A|X$_{1}$, X$_{2}$) = $\frac{\text{O(\ensuremath{X_{1}}, \ensuremath{X_{2}}, A) }}{\text{O}(X_{1},X_{2,}A_{T})}$
Donde A$_{T}$ en el denominador, indica el numero de estados para
todas las posibles energias de sistema.


\section*{2 Alemán 4}

2.1 Genitiv

Dessen mit Masculin

Dessen mit Feminin

Deren mit Neutrum

Und wie kann man einen Brief zum Arbeiten machen.
\end{document}
