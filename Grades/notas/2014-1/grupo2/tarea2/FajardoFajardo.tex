\documentclass[12pt]{article}
\usepackage[margin=1.0in]{geometry}
\usepackage[utf8]{inputenc}
\usepackage[T1]{fontenc}
\usepackage{lmodern}
\usepackage[spanish]{babel}
\usepackage{amsmath}

\begin{Documento}
%opening
\title{Laboratorio de Fisica}



Cuando la temperatura aumenta, tambien lo hace la energia interna de las moleculas , que como consecuencia puede hacer variar la distancia de separacion entre ellas.\\
La dilatacion termica de los objetos puede aproximarse dependiendo del caso: a la lineal y a la volumetrica.\\
Estas estan dadas por las siguientes ecuaciones:\\


\section{Ecuaciones}


\delta{L}=L\alpha\delta{T}
\alpha=\delta{L}/(L\delta{T})
A partir de esta ecuacion podemos hacer otra aproximacion de primer orden para obtener esto: \\
\delta{V}=V\beta\delta{T}

Donde $\beta$= 3$\alpha$.\\
\subsection{Sentido fisico}
Estas aproximaciones de primer orden o lineales, tienden a ser utiles pero su grado de validez esta restringido a la siguiente condicion:\\
\dT alpha\ll{T}
\end{document}
