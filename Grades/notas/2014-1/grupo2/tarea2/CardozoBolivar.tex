\documentclass{article}
\usepackage[utf8]{inputenc}
\usepackage{amsfonts}

\title{Cuaderno Estructural}
\author{David Cardozo}
\date{February 2014}

\begin{document}

\maketitle


\section{Seccion 1.6}
Sean $A,B,C,A',B' \subseteq U $ Como se comparan los siguientes conjuntos?

\subsection{ $(A \times B) \cup (C \times D)$ }
Sea $(x,y) \in ( A \times B) \cup (C \times D)$ luego $(x,y) \in A \times B$ o $(x,y) \in C \times D$ \\
Realizemos la prueba por casos: \\
Caso 1: \\ 
$ x \in A, y \in B$ Entonces como $A \subseteq A \cup C$ y $B \in B \cup D$, entonces $x \in A \cup C$ y $y \in B \cup D$. Entonces $(x,y) \in (A \cup C) \times (B \cup D)$. \\
Caso 2: \\
Similar "Left as an exercise to the reader"
\subsection{$(A \times B) \cap (C \times D)$ vs $(A \cap C) \times (B \cap D)$ }
Sea $(x,y) \in (A \times B) \cap (C \times D)$ \\
Entonces $(x,y) \in A \times B$ y $(x,y) \in C$ \\
es decir, $x \in A, y \in B, x \in C, y \in D.$ \\
ssi: $x \in (A \cap C)$ y $y \in B \cup D$ \\
ssi: $(x,y) \in (A \cap C) \times (B \cap D)$ \\
el si y solo si nos dimos cuenta por ser todos los pasos por "reversibles
\section{Para todo $ a \in \mathbb{R}$ }
\begin{eqnarray}
H_{a} = \left\{ x,y \in \mathbb{R}^2 : y=a \right\} \\
V_{a} = \left\{ x,y \in \mathbb{R}^2: x=a  \right\}
\end{eqnarray}
Dibuje en el plano. 
\[
B = \bigcup _{a \in [2,5]} (H_{a} \cap V{a})
\]
Entonces 
\[
B = \left\{ (x,y): x=y, x \in [2,5] \right\}
\]
\section{ Sea }
\[
A = \bigcup_{a \in [0,1)} ( \bigcap_{b \in (a,6]} [b,b+a))
\]
Para el caso particular, observemos
\[
\bigcap_{b \in ( \frac{1}{2} , 6]} [b,b) + \frac{1}{2} = \emptyset
\]
Entonces probemos que para todo $ a \in [0,1] $
\[
B = \bigcap_{b \in (a,b]} \lbrack b, b+a) = \emptyset
\]
Demostracion: 
Sea $a \in \lbrack 0, 1 \rbrack$. Entonces considere los intervalos $ \lbrack 1, 1+a \rbrack $ y $ \lbrack 6, 6+a \rbrack$ \\
Como $ 0 \leq a < 1$, entonces $ 1+a < 2 < 6 $, luego $ \lbrack 1, 1+a )  \cap \lbrack 6, 6+a \rbrack$ \\
No puede existir un $x$ que pertenezca a todo $ \rbrack b, b+a)$ para $ b \in  (a,6 \rbrack $ \\
o sea, $ b = \emptyset$
\section{Sea $A$ un conjunto}
Dado $ a \in A$, sea $E_{a} = { B \in B(A) : A \in B}$ \\
Demostrar
\[
\bigcup_{a \in A} E_{a} = P(a) \backslash {\emptyset}
\]
Demostracion: \\
Por doble contenencia, Sea 
\[
x \in \bigcup_{a \in A} E_{a}
\]
Entonces $X \in E_a $ para algun $ a \in A $. \\
Por definicion de $E_a, X \in P(A)$ y $ a \in X $ \\
Entonces $ x \neq \emptyset $ 
Luego $X \in P(A)$ y $X \not \in {\emptyset}$ entonces $X \in P(A) \backslash {\emptyset}$ \\
Parte 2 \\
Sea $ X \in P(A) \backslash {\emptyset}$. \\
Entonces $X \in P(A)$ Y $X \not \in  {\emptyset} $ \\
por lo tanto $X \neq \emptyset$. \\
Entonces existe un $ c \in X $  \\
Como $ X \in  P(A)$ y $c \in X$
entonces $X \in E_c$ 
Pero $ c \in A $, entonces $ c \in X$ y $X \in P(A)$ \\ 
o sea, $X \subseteq A$ \\
Luego, $X \in \bigcup_{a \in A} E_a$










\end{document}
