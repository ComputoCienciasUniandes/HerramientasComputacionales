\documentclass[12pt]{article}
\usepackage[margin=1.0in]{geometry}
\usepackage[utf8]{inputenc}
\usepackage[T1]{fontenc}
\usepackage{lmodern}
\usepackage[spanish]{babel}
\usepackage{amsmath}

\title{Notas de Cálculo Integral}

\begin{document}
\date{}
\maketitle


\section{Fórmula de Reducción del Seno}

\begin{large}
La integral de seno a la potencia n se puede reducir en integrales de menor potencia por medio de la fòrmula del seno, esta es:
\begin{equation}
 \int \sin^n x dx = -\frac{\cos x \sin^{n-1}x}{n} + \frac{n-1}{n}\int \sin^{n-2} xdx 
\end{equation}

\subsection{Integral definida de seno a la potencia n entre 0 y $\displaystyle \frac{pi}{2}$}
Con la fórmula de reducción del seno se puede simplificar la siguiente integral:

\begin{equation}
 \int_0^\frac{\pi}{2} \sin^n x dx = \frac{n-1}{n}\int_0^\frac{\pi}{2} \sin^{n-2} xdx 
\end{equation}

\subsubsection{Demostración}
Se utiliza el método de integración por partes de la siguiente manera en la fórmula de integración por partes:\\
\\
$\displaystyle u = \sin^{n-1} x $ \\
$\displaystyle du = (n-1)\sin^{n-2} x \cos xdx$\\
$\displaystyle dv = \sin xdx $ \\
$\displaystyle v = -\cos x $ \\


\begin{equation}
 \int udv = uv - \int vdu
\end{equation}

\begin{equation}
 \int \sin^n x dx = -\cos x\sin^{n-1} x + \int \cos x(n-1)\sin^{n-2} x \cos xdx
\end{equation}

\begin{equation}
 \int \sin^n x dx = -\cos x\sin^{n-1} x + (n-1)(\int \sin^{n-2} xdx - \int \sin^n xdx)
\end{equation}

\begin{equation}
 \int \sin^n x dx + (n-1)\int \sin^n xdx = -\cos x\sin^{n-1} x + (n-1)\int \sin^{n-2} xdx
\end{equation}

\begin{equation}
 \int \sin^n x dx = -\frac{\cos x\sin^{n-1} x}{n} + \frac{n-1}{n}\int \sin^{n-2} xdx
\end{equation}

El primer término evaluado para x = $\displaystyle \frac{pi}{2}$ y x = 0, da 0, QED.

\subsection{Integral definida de seno a la potencia de 2n entre 0 y $\displaystyle \frac{pi}{2}$}
Con el resultado anterior se puede mostrar:

\begin{equation}
 \int_0^\frac{\pi}{2} \sin^{2n} x dx = \frac{\pi}{2} \cdot \frac{1}{2} \cdot \frac{3}{4} \cdot \frac{5}{6} \cdot ... \cdot \frac{2n-1}{2n}
\end{equation}

\subsubsection{Demostración}
Se utiliza el método de induccion:\\
Primero se verifica para n = 1:\\

\begin{equation}
 \int_0^\frac{\pi}{2} \sin^{2} x dx = \frac{1}{2} \cdot \int_0^\frac{\pi}{2} 1-\cos (2x)dx = \frac{\pi}{2} \cdot \frac{1}{2}
\end{equation}
\\
Ahora se supone que es valido para m y se demuestra para m+1:

\begin{equation}
 \int_0^\frac{\pi}{2} \sin^{2m} x dx = \frac{\pi}{2} \cdot ... \cdot \frac{2m-1}{2m}
\end{equation}

\begin{equation}
 \int_0^\frac{\pi}{2} \sin^{2(m+1)} x dx = \int_0^\frac{\pi}{2} \sin^{2m+2} x dx = \frac{2m+2-1}{2m+2} \cdot \int_0^\frac{\pi}{2} \sin^{2m} x dx
\end{equation}

\begin{equation}
 \int_0^\frac{\pi}{2} \sin^{2(m+1)} x dx = \frac{2(m+1)-1}{2(m+1)} \cdot \frac{\pi}{2} \cdot ... \cdot \frac{2m-1}{2m} 
= \frac{\pi}{2} \cdot ... \cdot \frac{2m-1}{2m} \cdot \frac{2(m+1)-1}{2(m+1)}
\end{equation}
Y así queda demostrado

\end{large}

\end{document}
