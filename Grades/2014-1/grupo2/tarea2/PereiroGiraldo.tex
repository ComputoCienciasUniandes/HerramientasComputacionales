\documentclass[12pt]{article}
\usepackage[margin=1.0in]{geometry} 
\usepackage[utf8]{inputenc}
\usepackage[T1]{fontenc}
\title{Notas de clase Fisica II}
\usepackage{lmodern}


\usepackage[spanish]{babel}
\usepackage{amsmath}
\begin{document}
\maketitle
\section{Primera Ley de la Termodinámica}
\subsection{Ecuación primera ley}
\begin{equation}
Uf-Ui=Q-W
\end{equation}
\subsubsection{Variables}
Uf= Energía interna final del sistema\\
Ui= Energía interna inicial del sistema\\
Q= Cantidad de calor\\
W= Trabajo efectuado por el sistema o hacia el sistema\\

\begin{equation}
U=Ek+Im
\end{equation}
\subsubsection{Variables}
U= Energía interna del sistema\\
Ek= Energía cinética de las moléculas\\
Im= Energía de interacción entre las moléculas\\

\section{Procesos termodinamicos}
\subsection{Procesos isovolumétricos}
Proceso termodinámico con volumen constante, no hay trabajo, puesto que no hay cambio de trabajo.\\
\begin{equation}
 \Delta U=Q
\end{equation}
\subsection{Procesos isobáricos}
Proceso termodinámico a presión constante.
\begin{equation}
 \Delta U=Q-W
\end{equation}
\subsection{Procesos isotérmicos}
Proceso termodinámico sin cambio de temperatura, por lo tanto, no hay cambio en la energía interna.
\begin{equation}
 Q=W
\end{equation}
\subsection{Procesos adiabáticos}
Proceso termodinámico sin cambio de calor, por lo tanto, Q=0 y la el cambio de energía interna del sistema es igual al trabajo negativo.
\begin{equation}
 \Delta U=-W
\end{equation}
\subsection{Procesos cíclicos}
Proceso termodinámico en el que el estado inicial del sistema es igual al estado final del sistema, generalmente, se repite varias veces.
\section{Proceso isotérmico}
T cte \\dT=0\\ dU=0\\ 
Q=dW\\
dW= PdVol\\
\begin{equation}
 dW=PdVol
\end{equation}
\begin{equation}
W=\int_{Vi}^{Vf} PdV = nRT\int_{Vi}^{Vf} \frac{dV}{V}= nRT  ln  \frac{Vf}{Vi}
\end{equation}

\section{Proceso isobárico}
Como el cambio de T es el mismo Uf-Ui también es el mismo
\begin{equation}
 \Delta U1=\Delta U2
\end{equation}
\begin{equation}
 nCvdt = nCpdT -PdV
\end{equation}
Para un gas ideal PV=nRT
\begin{equation}
 PdV=nRdT
\end{equation}
\begin{equation}
 nCvdT= nCpdT + nRdT
\end{equation}
\begin{equation}
 Cv= Cp-R
\end{equation}
\end{document}
