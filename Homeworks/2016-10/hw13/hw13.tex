
%--------------------------------------------------------------------
%--------------------------------------------------------------------
% Formato para los talleres del curso de Métodos Computacionales
% Universidad de los Andes
% 2015-10
%--------------------------------------------------------------------
%--------------------------------------------------------------------

\documentclass[11pt,letterpaper]{exam}
\usepackage[utf8]{inputenc}
\usepackage[spanish]{babel}
\usepackage{graphicx}
\usepackage{enumerate}
\usepackage{tabularx}
\usepackage[absolute]{textpos} % Para poner una imagen completa en la portada
\usepackage{multirow}
\usepackage{float}
\usepackage{hyperref}
\usepackage{breakurl}

\decimalpoint
%\usepackage{pst-barcode}
%\usepackage{auto-pst-pdf}

\newcommand{\base}[1]{\underline{\hspace{#1}}}
\boxedpoints
\pointname{ pt}
%\extrawidth{0.75in}
%\extrafootheight{-0.5in}
\extraheadheight{-0.15in}
%\pagestyle{head}

%\noprintanswers
%\printanswers


\usepackage{upquote,textcomp}
\newcommand\upquote[1]{\textquotesingle#1\textquotesingle} % To fix straight quotes in verbatim

\begin{document}

\begin{center}

\includegraphics[width=16cm]{header.png}

\vspace{1.0cm}
{\Large Herramientas Computacionales \\
 Tarea 13} \\ 
 \medskip
\textsc{Semana 14: Métodos de Monte Carlo: Métodos de integración} \\
2016-I\\
\end{center}

\vspace{0.5cm}

{\Large Instrucciones de Entrega}\\

\noindent
La solución a este taller debe subirse por SICUA antes de terminar 
el horario de clase.
\noindent
Consiste de un IPython Notebook con el nombre
\verb"NombreApellido_hw13"
el cual debe contener todas las intrucciones necesarias del ejercicio.

\begin{questions}

\question[60]{\bf{La integración}} 

Escriba una función que haga la integral:

\begin{equation}
\int\displaylimits^1_0 \int\displaylimits^1_0 \ldots \int\displaylimits^1_0 (x_1+x_1+ \ldots + x_{10})^2 \, \mathrm{d} x_1 \mathrm{d} x_2 \ldots \mathrm{d} x_{10}
\end{equation}

Mediante el método de Sampling (mostrado en el video). Recuerde que la integral puede aproximarse eligiendo N vectores $\vec{x}=(x_1,x_2,\ldots,x_{10})$ aleatoriamente en el intervalo 

\[ ([a_1,b_1],[a_2,b_2],\ldots,[a_{10},b_{10}]) \]

(note que es algo fácil de hacer, simplemente eligiendo un punto aleatorio para cada dimensión), y evaluando:

\begin{equation}
I \approx \frac{(b_1-a_1) \times (b_2-a_2) \time \ldots (b_{10}-a_{10})}{N} \sum_{i=1}^N f(\vec{x}_i)
\end{equation}

Puede comparar su resultado con el valor de la integral calculado analíticamente, $I=155/6$

\question[40] {\bf{El error}} 

Ahora observemos el comportamiento del error para N=100 y N=10000. Calcule el valor de la integral unas 100 veces para cada caso y calcule la desviaci\'on est\'andar (la funci\'on \verb+std+ de Numpy hace eso). ¿Es razonable pensar que el error es proporcional a $1/\sqrt{N}$?

\end{questions}

\end{document}

