
%--------------------------------------------------------------------
%--------------------------------------------------------------------
% Formato para los talleres del curso de Métodos Computacionales
% Universidad de los Andes
% 2015, curso de vacaciones
%--------------------------------------------------------------------
%--------------------------------------------------------------------

\documentclass[11pt,letterpaper]{exam}
\usepackage[utf8]{inputenc}
\usepackage[spanish]{babel}
\usepackage{graphicx}
\usepackage{enumerate}
\usepackage{tabularx}
\usepackage[absolute]{textpos} % Para poner una imagen en posiciones arbitrarias
\usepackage{multirow}
\usepackage{float}
\usepackage{hyperref}
\usepackage{breakurl}
\decimalpoint


\begin{document}
\begin{center}

\includegraphics[width=16cm]{header.png}

\vspace{1.0cm}
{\Large Herramientas Computacionales \\
 Tarea 10} \\
\textsc{Semana 11 - Métodos numéricos: método de la bisección y de Newton-Raphson }\\
2016-I\\
\end{center}

%\begin{textblock*}{40mm}(10mm,20mm)
%  \includegraphics[width=3cm]{logoUniandes.png}
%\end{textblock*}

%\begin{textblock*}{40mm}(164mm,20mm)
%  \includegraphics[width=3cm]{logoUniandes.png}
%\end{textblock*}

\vspace{0.5cm}

{\Large Instrucciones de Entrega}\\

\noindent
La solución a este taller debe subirse por SICUA antes de terminar 
el horario de clase.
\noindent
Consiste de un IPython Notebook con el nombre
\verb"NombreApellido_hw10"
el cual debe contener todas las intrucciones necesarias del ejercicio.

\begin{questions}

\question[50] {\bf{Raíces de un número}} 

Encuentre la $\sqrt[7]{7} $ usando el método de la bisección con una precisión de 5 cifras significativas.

Note que esto es equivalente a encontrar la soluci\'on a 
\begin{equation}
x^{7} - 7 = 0
\end{equation}

\question[50]{\bf{Ceros de un polinomio}} 

Encuentre todas las soluciones a la ecuación $3x^5 + 20x^4 -10x^3 -240x^2 -250x +200 = 0$ usando el método de Newton-Raphson. Eso es equivalente a encontrar los ceros del polinomio. Recuerde que cada punto inicial lo llevará, en caso que haya convergencia, a un cero, por lo que necesita definir 5 puntos iniciales adecuados. Para eso haga primero una inspección visual que le permita encontrar buenos puntos iniciales para el algoritmo, y a partir de eso justifique sus elecciones.

\end{questions}

\end{document}
