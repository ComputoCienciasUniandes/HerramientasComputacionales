%--------------------------------------------------------------------
%--------------------------------------------------------------------
% Formato para los talleres del curso de Herramientas Computacionales
% Universidad de los Andes
%--------------------------------------------------------------------
%--------------------------------------------------------------------

\documentclass[11pt,letterpaper]{exam}
\usepackage[utf8]{inputenc}
\usepackage[spanish]{babel}
\usepackage{graphicx}
\usepackage{mdframed}
\usepackage{tabularx}
\usepackage[absolute]{textpos} % Para poner una imagen completa en la portada
\usepackage{multirow}
\mdfdefinestyle{mystyle}{leftmargin=1cm,rightmargin=1cm,linecolor=red}
\usepackage{float}
\usepackage{hyperref}
\decimalpoint
%\usepackage{pst-barcode}
%\usepackage{auto-pst-pdf}

\newcommand{\base}[1]{\underline{\hspace{#1}}}
\boxedpoints
\pointname{ pt}
%\extrawidth{0.75in}
%\extrafootheight{-0.5in}
\extraheadheight{-0.15in}
%\pagestyle{head}

%\noprintanswers
%\printanswers
\renewcommand{\solutiontitle}{}
\SolutionEmphasis{\color{blue}}

\usepackage{upquote,textcomp}
\newcommand\upquote[1]{\textquotesingle#1\textquotesingle} % To fix straight quotes in verbatim

\begin{document}
\begin{center}
{\Large Herramientas Computacionales} \\
Taller 8 - \textsc{Python}: histogramas \\
{\small \it Octubre de 2014}
\end{center}

\begin{textblock*}{40mm}(10mm,20mm)
  \includegraphics[width=3cm]{logoUniandes.pdf}
\end{textblock*}

\begin{textblock*}{40mm}(161mm,20mm)
  \includegraphics[width=3cm]{logoUniandes.pdf}
\end{textblock*}

\vspace{1cm}

La solución de este taller debe ser presentada en un solo archivo con nombre \verb+NombreApellido_HW8.ipynb+. En cada parte del ejercicio se entrega 1/3  de los puntos si el código propuesto es razonable, 1/3 si se puede ejecutar y 1/3 si entrega resultados correctos.
\begin{questions}

\question[100] {\bf{Terremotos}} 

El archivo \verb+quakes.csv+\footnote{Datos obtenidos del {\it International Seismological Centre}.} contiene información sobre los terremotos ocurridos entre el primero de enero de 2000 y el 31 de diciembre de 2005.

\begin{parts}
	\part[10] Graficar la ubicación de los terremotos, su latitud en función de su longitud.
\begin{figure}[h]
	\centering
		\includegraphics[width=0.8\textwidth]{map.png}
	\caption{Ubicación de los terremotos en la Tierra.}
	\label{fig: map}
\end{figure}

	\part[30] Hacer un histograma normalizado para las intensidades, de tal forma que se obtenga algo similar a lo mostrado en la figura \ref{fig: mag histogram}. Use la opción \verb+range=(1.999, 7.001)+ y utilice 50 {\it bins}.
	\part[10] Calcular un arreglo que contenga los tiempos de espera entre terremotos consecutivos. En el archivo \verb+quakes.csv+ la columna \verb+DATETIME+ contiene la cantidad de segundos transcurridos desde el inicio del año 1900 hasta el momento en el que el terremoto correspondiente se produjo, el archivo está organizado de acuerdo a esta columna.
	\part[10] Calcular el promedio $t^*$ de los tiempos de espera usando la función \verb+mean+ de \verb+numpy+.
	\part[30] Hacer el histograma normalizado para los tiempos de espera, 
	\part[10] y sobre los mismos ejes graficar la función $\frac{1}{t^*} e^{- t/t^*}$, donde $t$ es el tiempo de espera y $t^*$ el tiempo de espera promedio\footnote{Para más información ver \href{http://en.wikipedia.org/wiki/Exponential\_distribution}{http://en.wikipedia.org/wiki/Exponential\_distribution}}. Al final debe obtenerse algo similar a lo mostrado en la figura \ref{fig: histograma tiempo de espera}.
\end{parts}

\begin{figure}
\centering
\includegraphics[width=0.8\textwidth]{quakes.pdf}
\caption{Histograma para las intensidades de los terremotos.}
\label{fig: mag histogram}
\end{figure}

\begin{figure}
\centering
\includegraphics[width=0.8\textwidth]{espera.pdf}
\caption{Histograma para los tiempos de espera entre terremotos consecutivos.}
\label{fig: histograma tiempo de espera}
\end{figure}

\end{questions}
\end{document}
