% Herramientas Computacionales
% Universidad de los Andes
% 2015-1
% Taller 3

% Si no está instalado revtex eliminar la siguiente línea, usar la siguiente y eliminar la línea \affiliation
\documentclass[aps,twocolumn,letter,10pt,floatfix]{revtex4}
%\documentclass[twocolumn,letter]{article}
\usepackage[utf8]{inputenc}

\begin{document}

\title{Optimal States for Bell inequality Violations using Quadrature Phase Homodyne Measurements} 
\author{W. J. Munro}
\affiliation{Centre for Laser Science,Department of Physics, University of Queensland, QLD 4072, Brisbane, Australia}
\date{February 1, 2008}

\maketitle

\section{Introduction}
 
There has been recently active interest in tests of quantum mechanics \cite{epr} versus local realism in a high efficiency detection limit. Several authors \cite{apm,wg,bd} including ourselves have considered detection schemes quadrature phase homodyne measurements. Such schemes use strong local oscillators and hence have very high detection efficiency \cite{ejh}. This removes one of the current loopholes \cite{jmar,ppa,mpma,ets} and potentially allows a strong test of quantum mechanics to be performed.
 
The original idea of Gilchrist et. al. \cite{apm} was to use a circle or pair coherent state \cite{gs,kg,ml} produced by nondegenerate parametric oscillation with the pump mode adiabatically eliminated. Using highly efficient quadrature phase homodyne measurements, the Clauser Horne strong Bell inequality \cite{bell,jma,jfa} could be tested in an all optical regime. A small (approximately 1.5\%) but significant theoretical violation was found for this extremely ideal system. While the mean photon number for the system may be low (approximately 1.12), the use of homodyne measurements allow a macroscopic current to be detected.
 
 
\bibliographystyle{plain}.
 
\bibliography{referencias.bib}
 
\end{document}

