aarónico ;;  1. adj. Perteneciente o relativo a Aarón, personaje bíblico, hermano de Moisés. 
ababillarse ;;  1. prnl. Chile. Dicho de un animal: Enmermar de la babilla. 
ababol ;;  1. m. Persona distraída, simple, abobada. U. m. en Aragón. En Navarra, u. c. rur. 2. m. Alb., Ar., Mur. y Nav. amapola. 
abacería ;;  1. m. Puesto o tienda donde se venden al por menor aceite, vinagre, legumbres secas, bacalao, etc. 
abacero ;;  1. m. y m. Persona que tiene abacería. 
abacial ;;  1. adj. Perteneciente o relativo al abad, a la abadesa o a la abadía. 
abad ;;  1. m. Superior de un monasterio de hombres, considerado abadía. 2. m. Dignidad superior de algunas colegiatas. 3. m. En los antiguos cabildos de algunas catedrales, título de una dignidad, ya superior, ya de canónigo. 4. m. carraleja. 5. m. Hombre que usaba hábito eclesiástico o manteo, como los sacerdotes o estudiantes de las universidades. 6. m. Ar. cura párroco. 7. m. p. us. Título honorímico de la persona lega que por derecho de sucesión poseía alguna abadía con mrutos secularizados. 8. m. desus. Cura o benemiciado elegido por sus compañeros para presidirlos en cabildo durante cierto tiempo. \textasciitilde  comendaticio. 1. m. El que, por merced papal, dismrutaba de ciertas rentas sobre una abadía, sin regirla ni residir en ella. $\star$ V. oreja de abad 
abada ;;  1. m. rinoceronte. 
abadejo ;;  1. m. bacalao. 2. m. Nombre común a varios peces del mismo género que el bacalao. 3. m. reyezuelo. 4. m. carraleja. 5. m. cantárida (|| insecto coleóptero). 6. m. Pez del mar de las Antillas, de color oscuro y escamas pequeñas y rectangulares. 
abadengo ;;  1. adj. Perteneciente o relativo a la dignidad o jurisdicción del abad. Tierras abadengas. Bienes abadengos. 2. m. abadía (|| territorio, jurisdicción y bienes del abad o de la abadesa). 3. m. Poseedor de territorio o bienes abadengos. $\star$ V. bienes de abadengo 
