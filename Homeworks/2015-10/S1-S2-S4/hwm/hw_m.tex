
%--------------------------------------------------------------------
%--------------------------------------------------------------------
% Formato para los talleres del curso de Herramientas Computacionales
% Universidad de los Andes
% 2015-10
%--------------------------------------------------------------------
%--------------------------------------------------------------------

\documentclass[11pt,letterpaper]{exam}
\usepackage[utf8]{inputenc}
\usepackage[spanish]{babel}
\usepackage{graphicx}
\usepackage{mdframed}
\usepackage{tabularx}
\usepackage[absolute]{textpos} % Para poner una imagen completa en la portada
\usepackage{multirow}
\mdfdefinestyle{mystyle}{leftmargin=1cm,rightmargin=1cm,linecolor=red}
\usepackage{float}
\usepackage{hyperref}
\decimalpoint
%\usepackage{pst-barcode}
%\usepackage{auto-pst-pdf}

\newcommand{\base}[1]{\underline{\hspace{#1}}}
\boxedpoints
\pointname{ pt}
%\extrawidth{0.75in}
%\extrafootheight{-0.5in}
\extraheadheight{-0.15in}
%\pagestyle{head}

%\noprintanswers
%\printanswers
\renewcommand{\solutiontitle}{}
\SolutionEmphasis{\color{blue}}

\usepackage{upquote,textcomp}
\newcommand\upquote[1]{\textquotesingle#1\textquotesingle} % To fix straight quotes in verbatim

\begin{document}
\begin{center}
{\Large Herramientas Computacionales} \\
Taller M - \textsc{Mathematica}: documentación \\
{\it Febrero de 2015}
\end{center}

\begin{textblock*}{40mm}(10mm,20mm)
  \includegraphics[width=3cm]{logoUniandes.png}
\end{textblock*}

\begin{textblock*}{40mm}(161mm,20mm)
  \includegraphics[width=3cm]{logoUniandes.png}
\end{textblock*}

\vspace{0.5cm}

\vspace{0.5cm}

Esta tarea se debe entregar a través de Sicua plus en un cuaderno de {\it Mathematica} con el nombre del archivo en el formato \verb"NombreApellido_HWM.nb". \\

\vspace{0.1cm}

\begin{questions}

\question[100] {\bf Explorando {\it Mathematica}}

Explore la documentación de {\it Mathematica}, escoja 10 ejemplos que encuentre interesantes y reúnalos en un cuaderno.

\end{questions}

\end{document}
