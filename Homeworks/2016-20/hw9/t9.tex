\documentclass[11pt,letterpaper]{exam}
\usepackage{amsmath}
\usepackage[utf8]{inputenc}
\usepackage[spanish]{babel}
\usepackage{graphicx}
\usepackage{tabularx}
\usepackage[absolute]{textpos} % Para poner una imagen en posiciones arbitrarias
\usepackage{multirow}
\usepackage{float}
\usepackage{hyperref}
\usepackage{breakurl}
\decimalpoint

\begin{document}
\begin{center}

\includegraphics[width=16cm]{header.png}

\vspace{1.0cm}
{\Large Herramientas Computacionales - Tarea 9}\\
2016-II
\end{center}

%\begin{textblock*}{40mm}(10mm,20mm)
%  \includegraphics[width=3cm]{logoUniandes.png}
%\end{textblock*}

%\begin{textblock*}{40mm}(164mm,20mm)
%  \includegraphics[width=3cm]{logoUniandes.png}
%\end{textblock*}

\vspace{0.5cm}

\noindent
Los archivo del c\'odigo fuente debe subirse a Sicua plus en un \'unico archivo
\verb".zip" con el nombre del estudiante en el formato \verb"NombreApellido_hw9.zip" antes que termine la clase.


Considere un tiro parab\'olico desde una altura inicial \verb'y0 = 10' metros hasta una altura final \verb'yf = 0'. La velocidad inicial es de \verb'v0 = 4' m/s y el \'angulo \verb'theta' es un par\'ametro variable. El objetivo de este ejercicio es hallar el alcance m\'aximo en funci\'on del \'angulo de tiro. Recordemos que las ecuaciones para la altura \verb'y' y el dezplazamiento horizontal \verb'x' son

\begin{align}
y &= y_0 + v_0\sin\theta \cdot t -\frac{1}{2}gt^2\label{1}\\
x &= v_0\cos\theta \cdot t\label{2}
\end{align}

\vspace{0.5cm}

\begin{questions}
 

\question[1.0] {Funciones}

Implemente las funciones \verb'fy' y \verb'fx' que reciben como par\'ametros el tiempo y el \'angulo y retornan la posici\'on vertical (eq. \ref{1}) y horizontal (eq. \ref{2}). Respectivamente.

\question[1.5] {Newton-Raphson, una iteraci\'on}

Utilice el m\'etodo de Newton para hallar e imprimir el tiempo de vuelo y el alcance horizontal m\'aximo a un \'angulo de tiro de $\pi/4$.

\question[2.5] {Newton-Raphson, varias iteraciones}

Utilice m\'ultiples veces el m\'etodo de Newton para hallar el tiempo de vuelo y el alcance horizontal m\'aximo a $15$ \'angulos de tiro equidistantes entre $0$ y $\pi/2$ (no incluya los valores l\'imite). Almacene los \'angulos, tiempos y alcances m\'aximos en listas y grafique utilizando \verb'scatter' tiempo vs. \'angulo de tiro y alcance m\'aximo vs. \'angulo de tiro en gr\'aficas separadas.

\end{questions}

\end{document}
