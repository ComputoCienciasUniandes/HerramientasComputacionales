%--------------------------------------------------------------------
%--------------------------------------------------------------------
% Formato para los talleres del curso de Herramientas Computacionales
% Universidad de los Andes
%--------------------------------------------------------------------
%--------------------------------------------------------------------

\documentclass[11pt,letterpaper]{exam}
\usepackage[utf8]{inputenc}
\usepackage[spanish]{babel}
\usepackage{graphicx}
\usepackage{mdframed}
\usepackage{tabularx}
\usepackage[absolute]{textpos} % Para poner una imagen completa en la portada
\usepackage{multirow}
\mdfdefinestyle{mystyle}{leftmargin=1cm,rightmargin=1cm,linecolor=red}
\usepackage{float}
\usepackage{hyperref}
\decimalpoint
%\usepackage{pst-barcode}
%\usepackage{auto-pst-pdf}
\usepackage{amsmath}

\newcommand{\base}[1]{\underline{\hspace{#1}}}
\boxedpoints
\pointname{ pt}
%\extrawidth{0.75in}
%\extrafootheight{-0.5in}
\extraheadheight{-0.15in}
%\pagestyle{head}

%\noprintanswers
%\printanswers
\renewcommand{\solutiontitle}{}
\SolutionEmphasis{\color{blue}}

\usepackage{upquote,textcomp}
\newcommand\upquote[1]{\textquotesingle#1\textquotesingle} % To fix straight quotes in verbatim

\begin{document}
\begin{center}
{\Large Herramientas Computacionales} \\
Taller 13 - \textsc{Python}: Simulaciones de Monte-Carlo (Integrales) \\
{\small \it Octubre de 2014}
\end{center}

%\begin{textblock*}{40mm}(10mm,20mm)
%  \includegraphics[width=3cm]{logoUniandes.pdf}
%\end{textblock*}

%\begin{textblock*}{40mm}(161mm,20mm)
%  \includegraphics[width=3cm]{logoUniandes.pdf}
%\end{textblock*}

\vspace{1cm}

%La solución de este taller debe ser presentada en un solo archivo con nombre \verb+NombreApellido_HW13.ipynb+.
\begin{questions}

\question[100] {\bf{Integrales}}

Realizar las siguientes integrales usando el metodo de Monte-Carlo y usando la librer\'ia \\
 \verb+from scipy.integrate import quad+

\begin{parts}
	\part[20] $\int_{0}^{\pi} sin^2(\pi cos(\theta))cos^2(\theta)d\theta$
	\part[10] $\int_{0}^{\pi} sin^4(3x)dx$
	\part[20] $\int_{-4}^{4} \dfrac{e^{-x^2}}{((x-3)^2+0.01^2)}dx$, Discuta por qu\'e la integral usando el M\'etodo de Monte-Carlo no es adecuado
	para evaluar esta integral.
	\part[10] Investigue y explique brevemente que M\'etodo de Monte-Carlo se puede utilizar para realizar la integral
	del punto anterior.
	\part[15] $\int_{0}^{10} \dfrac{x^3}{x^4+16}dx$
	\part[25] $\int_{0}^{\infty} e^{-(x^2+y^2+z^2+w^2)}dxdydzdw$

\end{parts}



\end{questions}
\end{document}
