
%--------------------------------------------------------------------
%--------------------------------------------------------------------
% Formato para los talleres del curso de Métodos Computacionales
% Universidad de los Andes
% 2015-10
%--------------------------------------------------------------------
%--------------------------------------------------------------------

\documentclass[11pt,letterpaper]{exam}
\usepackage[utf8]{inputenc}
\usepackage[spanish]{babel}
\usepackage{graphicx}
\usepackage{enumerate}
\usepackage{tabularx}
\usepackage[absolute]{textpos} % Para poner una imagen completa en la portada
\usepackage{multirow}
\usepackage{float}
\usepackage{hyperref}
\usepackage{breakurl}

\decimalpoint
%\usepackage{pst-barcode}
%\usepackage{auto-pst-pdf}

\newcommand{\base}[1]{\underline{\hspace{#1}}}
\boxedpoints
\pointname{ pt}
%\extrawidth{0.75in}
%\extrafootheight{-0.5in}
\extraheadheight{-0.15in}
%\pagestyle{head}

%\noprintanswers
%\printanswers


\usepackage{upquote,textcomp}
\newcommand\upquote[1]{\textquotesingle#1\textquotesingle} % To fix straight quotes in verbatim

\begin{document}

\begin{center}
{\Large Herramientas Computacionales \\
 Taller 13}\\
Profesores: \\ Felipe G\'omez\\ Juan David Orjuela \\
Fecha de Publicación: {\small \it Noviembre 3 de 2015}\\
\end{center}

\begin{textblock*}{40mm}(10mm,20mm)
  \includegraphics[width=3cm]{logoUniandes.png}
\end{textblock*}

\begin{textblock*}{40mm}(161mm,20mm)
  \includegraphics[width=3cm]{logoUniandes.png}
\end{textblock*}

\vspace{0.5cm}

{\Large Instrucciones de Entrega}\\

\noindent
La solución a este taller debe subirse por SICUA antes de terminar 
el horario de clase.
\noindent
Consiste de un IPython Notebook con el nombre
\verb"NombreApellido_hw13"
el cual debe contener todas las intrucciones necesarias del ejercicio.

Es importante realizar estos pasos correctamente, ya que se calificará con un
script que asigna la nota 0.0 si los archivos no están correctamente nombrados.

\begin{questions}

\question[50]{\bf{Sumando números aleatorios}} 

Descargue el archivo de datos 
\href{https://github.com/ComputoCienciasUniandes/HerramientasComputacionales/blob/master/Lectures/98.Python/EjercicioHerramientasEstadisticas.ipynb}{EjercicioHerramientasEstadisticas.ipynb}.

Este cuaderno de IPython tiene 4 funciones, para la primera parte use las dos primeras: \textit{lanzamientos} llama \textit{RanSum}. Su finalidad es generar N números aleatorios que luego son sumados para ser guardados en una lista, haciendo el experimento M veces, y luego hacer un histograma con esos resultados. Haga una gráfica y un ajuste de los parámetros. ¿De qué tipo de distribución de probabilidad se trata? ¿Qué parámetros tiene?

\question[50] {\bf{Lanzamiento de una moneda}} 

Similarmente, \textit{realizaciones} llama \textit{coinflip}. En este caso, se lanza una moneda N veces y se guarda la frecuencia relativa con que sale \textit{cara}, luego se hace el experimento M veces y se guardan los resultados. Haga una gráfica y un ajuste de los parámetros. ¿De qué tipo de distribución de probabilidad se trata? ¿Qué parámetros tiene? ¿Es igual a la anterior? Si se diferencia, ¿en qué se diferencia y cuál puede ser la causa?

\end{questions}

\end{document}

