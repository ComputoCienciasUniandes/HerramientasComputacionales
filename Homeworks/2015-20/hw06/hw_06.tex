
%--------------------------------------------------------------------
%--------------------------------------------------------------------
% Formato para los talleres del curso de Métodos Computacionales
% Universidad de los Andes
% 2015-10
%--------------------------------------------------------------------
%--------------------------------------------------------------------

\documentclass[11pt,letterpaper]{exam}
\usepackage[utf8]{inputenc}
\usepackage[spanish]{babel}
\usepackage{graphicx}
\usepackage{enumerate}
\usepackage{tabularx}
\usepackage[absolute]{textpos} % Para poner una imagen completa en la portada
\usepackage{multirow}
\usepackage{float}
\usepackage{hyperref}
\decimalpoint
%\usepackage{pst-barcode}
%\usepackage{auto-pst-pdf}

\newcommand{\base}[1]{\underline{\hspace{#1}}}
\boxedpoints
\pointname{ pt}
%\extrawidth{0.75in}
%\extrafootheight{-0.5in}
\extraheadheight{-0.15in}
%\pagestyle{head}

%\noprintanswers
%\printanswers


\usepackage{upquote,textcomp}
\newcommand\upquote[1]{\textquotesingle#1\textquotesingle} % To fix straight quotes in verbatim

\begin{document}
\begin{center}
{\Large Herramientas Computacionales} \\
Taller 6 \\
Profesor: Felipe G\'omez\\
Fecha de Publicación: {\small \it Septiembre 8 de 2015}\\
\end{center}

\begin{textblock*}{40mm}(10mm,20mm)
  \includegraphics[width=3cm]{logoUniandes.png}
\end{textblock*}

\begin{textblock*}{40mm}(161mm,20mm)
  \includegraphics[width=3cm]{logoUniandes.png}
\end{textblock*}

\vspace{0.5cm}

{\Large Instrucciones de Entrega}\\

\noindent
La solución a este taller debe subirse por SICUA antes de terminar 
el horario de clase.
\noindent
Primero debe crearse una carpeta de trabajo llamada \verb"NombreApellido_hw6"
dentro de la cual se deben crear los archivos \verb"fibonacci.py" y
\verb"euclides.py". Una vez haya terminado de trabajar, comprimir la carpeta
desde la consola con el comando:

\verb"zip -r NombreApellido_hw6.zip NombreApellido_hw6"

\noindent Enviar el archivo comprimido \verb"NombreApellido_hw6.zip" por SICUA. 
Es importante realizar estos pasos correctamente, ya que se calificará con un
script que asigna la nota 0.0 si los archivos no están correctamente nombrados.




\begin{questions}

\question[50] {\bf{Serie de Fibonacci}} En un archivo llamado \verb"fibonacci.py" 
escribir una función llamada \verb"fib()" que tenga como entrada un número entero 
positivo $n$, como salida retorne el n-\'esimo t\'ermino de la serie de Fibonacci.

La función debe tener estructura iterativa.

\medskip
La serie $F_n$ tiene las siguientes propiedades:
\begin{enumerate}[I.)]
	\item $F_0 = 0$
	\item $F_1 = 1$
	\item $F_n = F_{n-1} +F_{n-2}$
\end{enumerate}

\question[50] {\bf{Algoritmo de Euclides para hallar MCD}} 
En un archivo llamado \verb"euclides.py"  escribir una función llamada \verb"mcd()"
que tenga como entrada dos números enteros, como salida retorne el máximo común divisor
calculado con el algoritmo de Euclides y que tenga estructura iterativa.

\medskip
\textit{``The Euclidean algorithm is the granddaddy of all algorithms, because is
the oldest nontrivial algorithm that has survived to the present day.'' - Donald
Knuth}.
\medskip

El algoritmo de Euclides para encontrar el máximo común divisor entre dos 
enteros positivos, $m$ y $n$, con la condición $m>=n$, puede escribirse así:

\begin{enumerate}[I.)]
	\item Dividir $m$ entre $n$ y tomar $r$ como el residuo de la división.

	\item Si $r$ es cero, $n$ es la respuesta; si no, continuar al siguiente paso.

	\item Hacer $m=n$ y $n=r$. Volver al paso 1.
\end{enumerate}

\end{questions}


\end{document}
