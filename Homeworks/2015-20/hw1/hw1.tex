
%--------------------------------------------------------------------
%--------------------------------------------------------------------
% Formato para los talleres del curso de Métodos Computacionales
% Universidad de los Andes
% 2015, curso de vacaciones
%--------------------------------------------------------------------
%--------------------------------------------------------------------

\documentclass[11pt,letterpaper]{exam}
\usepackage[utf8]{inputenc}
\usepackage[spanish]{babel}
\usepackage{graphicx}
\usepackage{tabularx}
\usepackage[absolute]{textpos} % Para poner una imagen en posiciones arbitrarias
\usepackage{multirow}
\usepackage{float}
\usepackage{hyperref}
\decimalpoint

\begin{document}
\begin{center}

\includegraphics[width=16cm]{header.png}

{\Large Herramientas Computacionales} \\
Tarea 1 - \textsc{Comandos b\'asicos de UNIX}\\
2015-Ago-04.\\
\end{center}

%\begin{textblock*}{40mm}(10mm,20mm)
%  \includegraphics[width=3cm]{logoUniandes.png}
%\end{textblock*}

%\begin{textblock*}{40mm}(164mm,20mm)
%  \includegraphics[width=3cm]{logoUniandes.png}
%\end{textblock*}

\vspace{0.5cm}

\noindent
La solución a este taller debe subirse por SICUA antes de terminar la clase.
\noindent
Los dos archivos c\'odigo fuente deben subirse en un \'unico archivo
\verb".zip" con el nombre \verb"NombreApellido_hw1.zip", por ejemplo
yo deber\'ia subir el zip \verb"FelipeGomez_hw1.zip".

\vspace{0.5cm}

\begin{questions}
 
\question[(40)] {\bf{Estudiantes de Mec\'anica :)}}

Escriba un script llamado \verb"mecanica.sh" que descargue el archivo de notas,
genere un archivo 
(\verb"mecanica.txt") en el que se escriban nombre, apellido y nota definitiva
de los estudiantes de mecánica que pasaron la materia.


\question[60]{\bf{Sort}}

Escriba un script llamado \verb"definitivas.sh" que descarge el archivo de notas, genere un listado 
(\verb"definitivas.txt") con las notas de todos los estudiantes que pasaron 
la materia, organizadas de mayor a menor usando el comando \verb"sort".



\end{questions}

\end{document}
