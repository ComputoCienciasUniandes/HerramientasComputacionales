
%--------------------------------------------------------------------
%--------------------------------------------------------------------
% Formato para los talleres del curso de Métodos Computacionales
% Universidad de los Andes
% 2015-10
%--------------------------------------------------------------------
%--------------------------------------------------------------------

\documentclass[11pt,letterpaper]{exam}
\usepackage[utf8]{inputenc}
\usepackage[spanish]{babel}
\usepackage{graphicx}
\usepackage{enumerate}
\usepackage{tabularx}
\usepackage[absolute]{textpos} % Para poner una imagen completa en la portada
\usepackage{multirow}
\usepackage{float}
\usepackage{hyperref}
\decimalpoint
%\usepackage{pst-barcode}
%\usepackage{auto-pst-pdf}

\newcommand{\base}[1]{\underline{\hspace{#1}}}
\boxedpoints
\pointname{ pt}
%\extrawidth{0.75in}
%\extrafootheight{-0.5in}
\extraheadheight{-0.15in}
%\pagestyle{head}

%\noprintanswers
%\printanswers


\usepackage{upquote,textcomp}
\newcommand\upquote[1]{\textquotesingle#1\textquotesingle} % To fix straight quotes in verbatim

\begin{document}
\begin{center}
{\Large Herramientas Computacionales} \\
Taller 7 \\
Profesor: Felipe G\'omez\\
Fecha de Publicación: {\small \it Septiembre 15 de 2015}\\
\end{center}

\begin{textblock*}{40mm}(10mm,20mm)
  \includegraphics[width=3cm]{logoUniandes.png}
\end{textblock*}

\begin{textblock*}{40mm}(161mm,20mm)
  \includegraphics[width=3cm]{logoUniandes.png}
\end{textblock*}

\vspace{0.5cm}

{\Large Instrucciones de Entrega}\\

\noindent
La solución a este taller debe subirse por SICUA antes de terminar 
el horario de clase.
\noindent
Primero debe crearse una carpeta de trabajo llamada \verb"NombreApellido_hw7"
dentro de la cual deben estar los archivos \verb"resorte.py" y una gráfica. 
Una vez haya terminado de trabajar, comprimir la carpeta
desde la consola con el comando:

\verb"zip -r NombreApellido_hw7.zip NombreApellido_hw7"

\noindent Enviar el archivo comprimido \verb"NombreApellido_hw7.zip" por SICUA. 
Es importante realizar estos pasos correctamente, ya que se calificará con un
script que asigna la nota 0.0 si los archivos no están correctamente nombrados.




\begin{questions}

\question[50] {\bf{Ley de Hooke I.}} En este ejercicio estudiaremos el movimiento 
armónico simple en un sistema masa-resorte. Para eso trabajaremos en el archivo 
llamado \verb"resorte.py" donde crearán una clase de objetos llamada \verb"masa",
esta masa debe tener los siguientes atributos:
\begin{itemize}
	\item Posición $x$
	\item Velocidad $V_x$
	\item Aceleración $a_x$
	\item Constante de resorte $k$
\end{itemize}

La clase \verb"masa" debe tener los métodos:

\begin{itemize}
	\item  \verb"__init__": Carga las condiciones iniciales del objeto.
	\item  \verb"CalculaFuerza": Utiliza la ley de Hooke para calcular la fuerza
	como una función que depende de $x$ de la forma $ F = -kx$
	\item  \verb"Muevete": Utilizando el método de Euler, actualiza la posición y la velocidad en cada $\Delta t$.
	\item  \verb"Imprime": Imprime el instante $t$ y la posición $x$.
\end{itemize}



\question[50]{\bf{Gráfica}} 
El script debe iniciarse con una masa de 0.200kg en la posición 0.01m, en reposo y 
un resorte con constante $k=0.5N/m$. Debe oscilar durante 5.0s. Los datos pueden redirigirse
hacia un archivo \verb"trayectoria.dat" y estos deben graficarse en un archivo \verb"trayectoria.png". También es válido (sólo por esta clase) trabajar en LibreOffice y
entregar un archivo \verb"trayectoria.ods" o \verb"trayectoria.xls"



\end{questions}


\end{document}
