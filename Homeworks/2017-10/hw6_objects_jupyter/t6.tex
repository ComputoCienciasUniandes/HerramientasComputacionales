\documentclass[11pt,letterpaper]{exam}
\usepackage{amsmath}
\usepackage[utf8]{inputenc}
\usepackage[spanish]{babel}
\usepackage{graphicx}
\usepackage{tabularx}
\usepackage[absolute]{textpos} % Para poner una imagen en posiciones arbitrarias
\usepackage{multirow}
\usepackage{float}
\usepackage{hyperref}
\usepackage{breakurl}
\decimalpoint

\begin{document}
\begin{center}

\includegraphics[width=16cm]{header.png}

\vspace{1.0cm}
{\Large Herramientas Computacionales - Tarea 6} \\
\textsc{Semana 8 - Programaci\'on Orientada a Objetos y Cuadernos de Jupyter}\\
2017-I\\
\end{center}

%\begin{textblock*}{40mm}(10mm,20mm)
%  \includegraphics[width=3cm]{logoUniandes.png}
%\end{textblock*}

%\begin{textblock*}{40mm}(164mm,20mm)
%  \includegraphics[width=3cm]{logoUniandes.png}
%\end{textblock*}

\vspace{0.5cm}

\noindent
Los archivo del c\'odigo fuente debe subirse a Sicua plus en un \'unico archivo
\verb".ipynb" con el nombre del estudiante en el formato \verb"NombreApellido_hw6.ipynb" antes que termine la clase.

El ejercicio debe ser resuelto en un notebook de Jupyter (\verb'.ipynb').
	
El objetivo de este ejercicio es implementar la clase \verb'Pendulo' de manera similar a como se implementa la clase \verb'Balon' en el video.

No olvide escribir \verb'%pylab inline' al comienzo de su notebook para visualizar.

\vspace{0.5cm}

\begin{questions}
 
\question[2.5] {\bf{Implementaci\'on de la clase \verb"Pendulo"}}

Implemente la clase \verb"Pendulo" con las siguientes caracter\'isticas

\begin{itemize}
\item El constructor \verb'__init__' recibe como par\'ametros \verb'x0, v0, l0': la posici\'on, la velocidad inicial y la longitud de la cuerda.

\item Los atributos incializados en el constructor son la posici\'on y velocidad actuales \verb'x, v', la longitud \verb'l' y las \textbf{listas} de tiempo, posici\'on y velocidad \verb'T, X, V'.

\item Los atributos mencionados deben ser correctamente inicializados en el constructor. All\'i mismo debe ser asignado el primer elemento de \verb'T' a \verb'0', y los primeros elementos de las dem\'as listas seg\'un las condiciones iniciales.

\item Escribir el m\'etodo \verb'calculaAceleracion', el cual calcula la aceleraci\'on del p\'endulo y la guarda como un atributo \verb'a'. La aceleraci\'on de un p\'endulo est\'a dada por 

$$ a = - \frac{g}{l} x $$


\item Escribir el m\'etodo \verb'muevete' an\'alogamente a como estaba en el video. Recuerde que los cambios en velocidad y posici\'on se pueden aproximar como $dv \approx a\cdot dt$ y $dx \approx v\cdot dt$.

\item El m\'etodo \verb'imprime' ahora se llamar\'a \verb'guarda' y ahora debe guardar los valores de \verb't, x, v' en las respectivas listas. Puede usar la funci\'on \verb'append' para hacerlo.

\end{itemize}
\question[1.5] {\bf{Creaci\'on del objeto y evoluci\'on}}

Cree un objeto de la clase \verb'Pendulo' y desarrolle la evoluci\'on similarmente a como se hizo en el video de preparaci\'on para un \verb'Deltat = 0.01' hasta un tiempo final de \verb'12.0'. Luego del ciclo las listas deben contener todos los valores de tiempos, posiciones y velocidades en el intervalo de tiempo considerado.

\question[1.0] {\bf{Gr\'afica}}

Realice una gr\'afica de $X$ contra $T$ utilizando la misma sintaxis del video de preparaci\'on de tal forma que se vea el comportamiento esperado del p\'endulo.

\end{questions}

\end{document}
