\documentclass[11pt,letterpaper]{exam}
\usepackage{amsmath}
\usepackage[utf8]{inputenc}
\usepackage[spanish]{babel}
\usepackage{graphicx}
\usepackage{tabularx}
\usepackage[absolute]{textpos} % Para poner una imagen en posiciones arbitrarias
\usepackage{multirow}
\usepackage{float}
\usepackage{hyperref}
\usepackage{breakurl}
\decimalpoint

\begin{document}
\begin{center}

\includegraphics[width=16cm]{header.png}

\vspace{1.0cm}
{\Large Herramientas Computacionales - Tarea 1} \\
\textsc{Semana 2 - UNIX}\\
2017-I\\
\end{center}

%\begin{textblock*}{40mm}(10mm,20mm)
%  \includegraphics[width=3cm]{logoUniandes.png}
%\end{textblock*}

%\begin{textblock*}{40mm}(164mm,20mm)
%  \includegraphics[width=3cm]{logoUniandes.png}
%\end{textblock*}

\vspace{0.5cm}

\noindent
El archivo del c\'odigo fuente debe subirse a Sicua plus en un \'unico archivo \verb'.sh' con el nombre del estudiante en el formato \verb"NombreApellido.sh" antes que termine la clase.

El enlace del archivo que vamos a utilizar es:

\url{https://raw.githubusercontent.com/ComputoCienciasUniandes/MetodosComputacionalesDatos/master/homework/2014-20/hw_1/notas_fisicaII_201320.dat}

El archivo contiene las notas de un curso de F\'isica 2. Las primeras tres columnas son las notas de los parciales, la cuarta columna es la nota del final y la sexta columna es la nota definitiva. Todas las notas se encuentran sobre 100. Una nota aprobatoria corresponde a 60 puntos o m\'as.

\vspace{0.5cm}

\begin{questions}
 
\question[2.5]

Imprima el string \verb'1.', luego imprima el n\'umero de estudiantes que ganaron el primero, segundo y tercer parcial. La salida se debe ver de la siguiente forma:

\begin{verbatim}
1.
n11
estudiantes ganaron el primer parcial.
n12
estudiantes ganaron el segundo parcial.
n13
estudiantes ganaron el tercer parcial.
\end{verbatim}

donde \verb'n11', \verb'n12' y \verb'n13' corresponden a las cantidades a imprimir.

\question[2.0]

Imprima el string \verb'2.', luego imprima el n\'umero de estudiantes que perdieron exactamente un parcial y ganaron la materia. La salida debe verse de la siguiente forma:

\begin{verbatim}
2.
n2
estudiantes perdieron exactamente un parcial y ganaron la materia.
\end{verbatim}

donde \verb'n2' corresponde a la cantidad solicitada.

\question[0.5]

Incluya en el script los comandos necesarios para eliminar el archivo descargado y todos los archivos auxiliares.

\end{questions}

\end{document}
