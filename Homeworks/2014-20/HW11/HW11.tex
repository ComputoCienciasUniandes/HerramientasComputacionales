%--------------------------------------------------------------------
%--------------------------------------------------------------------
% Formato para los talleres del curso de Herramientas Computacionales
% Universidad de los Andes
%--------------------------------------------------------------------
%--------------------------------------------------------------------

\documentclass[11pt,letterpaper]{exam}
\usepackage[utf8]{inputenc}
\usepackage[spanish]{babel}
\usepackage{graphicx}
\usepackage{mdframed}
\usepackage{tabularx}
\usepackage[absolute]{textpos} % Para poner una imagen completa en la portada
\usepackage{multirow}
\mdfdefinestyle{mystyle}{leftmargin=1cm,rightmargin=1cm,linecolor=red}
\usepackage{float}
\usepackage{hyperref}
\decimalpoint
%\usepackage{pst-barcode}
%\usepackage{auto-pst-pdf}

\newcommand{\base}[1]{\underline{\hspace{#1}}}
\boxedpoints
\pointname{ pt}
%\extrawidth{0.75in}
%\extrafootheight{-0.5in}
\extraheadheight{-0.15in}
%\pagestyle{head}

%\noprintanswers
%\printanswers
\renewcommand{\solutiontitle}{}
\SolutionEmphasis{\color{blue}}

\usepackage{upquote,textcomp}
\newcommand\upquote[1]{\textquotesingle#1\textquotesingle} % To fix straight quotes in verbatim

\begin{document}
\begin{center}
{\Large Herramientas Computacionales} \\
Taller 11 - \textsc{Python}: Distribuci\'on de Poisson \& Binomial \\
{\small \it Octubre de 2014}
\end{center}

%\begin{textblock*}{40mm}(10mm,20mm)
%  \includegraphics[width=3cm]{logoUniandes.pdf}
%\end{textblock*}

%\begin{textblock*}{40mm}(161mm,20mm)
%  \includegraphics[width=3cm]{logoUniandes.pdf}
%\end{textblock*}

\vspace{1cm}

%La solución de este taller debe ser presentada en un solo archivo con nombre \verb+NombreApellido_HW11.tar+.
En cada parte del ejercicio se entrega 1/3  de los puntos si el código propuesto es razonable, 1/3 si se puede ejecutar y 1/3 si entrega resultados correctos.
\begin{questions}

\question[100] {\bf{Poisson}}

Imaginemos que durante un dia lluvioso dise\~namos una superficie plana de $100m^2$, en dicha superficie se guardan los datos de la posici\'on de cada
gota de agua que cae sobre esta. Despues de 30 minutos han caido $N_{gotas} = 100000$ gotas de forma aleatoria sobre la superficie.

\begin{parts}
	\part[10] Escriba una funci\'on que simule la superficie con las posiciones de las usando $N_{gotas} = 100000$ gotas.
	\part[65] Escriba una funci\'on que cuente el n\'umero de gotas que hay en una superficie mas peque\~na de dimensiones $a \times a$
	donde $a<10m$, esta superficie se pone de forma aleatoria sobre diferentes lugares de la superficie grande.
	Por tanto esta funci\'on debe arrojar el n\'umero $N$ de gotas que hay en $P$ diferentes partes de la superficie grande.
	$a$ y $P$ deben ser	par\'ametros que entren por consola. Nota: Si la superficie peque\~na se sale de la superficie grande, el area de
	la superficie que se sale debe aparecer al lado opuesto de la superficie grande, es decir con condiciones de frontera periodicas.
		\part[15] Haga un histograma del n\'umero de gotas $N$ sobre la superficie. Que distribucion es esta?, Haga un fit de dicha distribucion
		(Se calificar\'a la estetica de la gr\'afica).
\end{parts}



\end{questions}
\end{document}
