\documentclass{beamer}
\usetheme{AnnArbor}
\usepackage[spanish]{babel}
\usepackage{lmodern}
\usepackage[T1]{fontenc}
\usepackage[utf8]{inputenc}
\usepackage{amsmath}
\usepackage{amsfonts}
\usepackage{amssymb}
\usepackage{graphicx}
\usepackage{hyperref}
\author{Prof. Sebastian Saiibi \& David Cardozo\inst{1}}
% - Give the names in the same order as the appear in the paper.
% - Use the \inst{?} command only if the authors have different
%   affiliation.
\title{Introducción a \LaTeXe }
\subtitle{Practicas, Hojas de Vida, Reprotes de Laboratorio} % A subtitle is optional and this may be deleted
%\logo{\includegraphics[height=0.8cm]{universidaddelosandes.png}\vspace{220pt}} 
\logo{\includegraphics[height=0.8cm]{universidaddelosandesciencias.png}}
%\logo{\includegraphics[height=0.8cm]{universidaddelosandescolombia.png}
\institute[Universidad de los Andes]
{
	\inst{1}%
	Física   \\
	Lectura $1$ Herramientas Computacionales \\
	Universidad de los Andes
	}
\date{\today} % - Either use conference name or its abbreviation.
\subject{PDF Information} % This is only inserted into the PDF information catalog. Can be left out. 
%\setbeamercovered{transparent}
%\setbeamertemplate{navigation symbols}{}

\begin{document}
	\maketitle
	
	\begin{frame}
		\frametitle{Introducíon a Herramientas}
		 \begin{block}
		 	\centering
		 	{Para esta clase utilizaremos unas notas escritas Alaminos, Martí y Merí  }
		 \end{block}
		 Este formato de presentación en \LaTeX{}  puede ser utilizado para futuras presentaciones. (Cultura del compartir y dar crédito ) 
		 \[ \frac{\partial y}{\partial x} = 3xy \]
	\end{frame}
	\begin{frame}
		\frametitle{Herramientas para la clase de hoy}
		\begin{block}
			\centering
			{Links importantes }
		\end{block}
		\begin{itemize}
			\item \href{https://www.sharelatex.com/}{Sharelatex}
			\item \href{http://detexify.kirelabs.org/classify.html	}{detexify}
			\item \href{http://tex.stackexchange.com/}{Tex StackExchange}
			\item \href{https://github.com/Davidnet/Uniandes-Beamer	}{Proyecto de Beamer}
			
		\end{itemize}
	
	\end{frame}
	
	
\end{document}