\documentclass[]{beamer}
\usefonttheme[onlymath]{serif}
\usetheme{AnnArbor}
\usepackage[spanish]{babel}
\usepackage{lmodern}
\usepackage[T1]{fontenc}
\usepackage[utf8]{inputenc}
\usepackage{amsmath}
\usepackage{amsfonts}
\usepackage{amssymb}
\usepackage{graphicx}
\usepackage{hyperref}
\usepackage{listings}
\usepackage{color}
\lstset{ escapeinside={\%*}{*)},  keywordstyle=\color{blue}} 
\spanishplainpercent

\setbeamerfont{author in head/foot}{size={\fontsize{4pt}{5pt}\selectfont}}
\author{Prof. Sebastian Saaibi \& David Cardozo\inst{1}}
% - Give the names in the same order as the appear in the paper.
% - Use the \inst{?} command only if the authors have different
%   affiliation.
\title{Introducción a Python }
\subtitle{¿Que es Python?, lenguajes de programación, ¡Hola Mundo!, variables, condicionales y lectura de archivos  } % A subtitle is optional and this may be deleted
%\logo{\includegraphics[height=0.8cm]{universidaddelosandes.png}\vspace{220pt}} 
\logo{\includegraphics[height=0.8cm]{universidaddelosandesciencias.png}}
%\logo{\includegraphics[height=0.8cm]{universidaddelosandescolombia.png}
\institute[Universidad de los Andes]
{
	\inst{1}%
	Física   \\
	Lectura $3$ Herramientas Computacionales \\
	Universidad de los Andes
}
\date{\today} % - Either use conference name or its abbreviation.
\subject{Introducción a Python} % This is only inserted into the PDF information catalog. Can be left out. 
%\setbeamercovered{transparent}
%\setbeamertemplate{navigation symbols}{}

\begin{document}
	
\definecolor{keywords}{RGB}{255,0,90}
\definecolor{comments}{RGB}{0,0,113}
\definecolor{red}{RGB}{160,0,0}
\definecolor{green}{RGB}{0,150,0}
\maketitle



\begin{frame}{Aprendiendo Python de la manera más difícil}
	Existen muchas investigaciones en el ámbito de como enseñar un lenguaje de programación. En general un programador realiza las siguientes 3 actividades cuando quiere aprender un lenguaje de programación:
\begin{itemize}[<+->]
	\item Seguir paso a paso cada ejercicio y ejemplo.
	\item Escribir cada ejemplo directamente
	\item Ejecutar
\end{itemize}

\pause
Es una actividad \alert{dificil}, pues requiere ser un estudiante activo.

\end{frame}

\begin{frame}
	\frametitle{Introdución (Continuación \ldots)}
	Es objetivo de esta clase, enseñar 3 habilidades esenciales que todo programador debe tener:
	\begin{itemize}[<+->]
		\item Leer y Escribir
		
		Aprender a escribir los símbolos raros que aparecen en las muestras de código $ \backslash, \&, \%, /, == $
		\item Atención a los detalles.
		
		La separación entre buenos y malos programadores es cuan atentos son a los detalles, con esta habilidad evitas \alert{bugs}.
		\item Encontrar las diferencias.
		
		Esta una habilidad que se desarrolla con el tiempo, ver ejemplos de códigos muy similares  y encontrar las diferencias. 
	\end{itemize}
	
\end{frame}

\begin{frame}{Lo mas importante}
	\begin{columns}[T] % contents are top vertically aligned
		\begin{column}[T]{5cm} % each column can also be its own environment
		\huge No hacer \alert{``copiar y pegar''} sin entender cada linea de código.
		\end{column}
		\begin{column}[T]{5cm} % alternative top-align that's better for graphics
			\centering
			\includegraphics[height=4.5cm]{codereview.jpg}
		\end{column}
	\end{columns}
\end{frame}

\begin{frame}[fragile]
	\frametitle{Características generales}
	
	\begin{block}{Lenguaje en sí (ver \emph{import this})}
		\begin{itemize}
			\item Muy alto nivel
			\item Sintaxis uniforme y sencilla
			\item Completamente orientado a objetos
			\item Multiparadigma
			\item Énfasis está en la eficiencia de desarrollo
		\end{itemize}
	\end{block}
	\begin{lstlisting}[language=Python,inputencoding=utf8, 
	basicstyle=\ttfamily\small, 
	keywordstyle=\color{keywords},
	commentstyle=\color{comments},
	stringstyle=\color{red},
	showstringspaces=false,
	identifierstyle=\color{green}, basicstyle=\tiny]
	# -*- encoding: utf-8 -*-
	
	"""
	>>> import this # doctest: +NORMALIZE_WHITESPACE, +ELLIPSIS
	The Zen of Python, by Tim Peters
	<BLANKLINE>
	Beautiful is better than ugly.
	Explicit is better than implicit.
	Simple is better than complex.
	Complex is better than complicated.
	...
	"""
	\end{lstlisting}
\end{frame}

\begin{frame}{La configuración tipica en diferentes sistemas operativos.}
	\begin{itemize}
		\item En Windows: \alert{Powershell.}
		\item En Mac: \alert{Terminal}
		\item En Linux: \alert{bash.}
	\end{itemize}
\end{frame}

\begin{frame}[fragile]
	\frametitle{}
	
	\begin{lstlisting}[language=bash,caption=Bash y Revision de Python]
	$ python
	Python 2.6.5 (r265:79063, Apr  1 2010, 05:28:39)
	[GCC 4.4.3 20100316 (prerelease)] on linux2
	Type "help", "copyright", "credits" or "license"
	
	for more information.
	>>>
	$ mkdir clase
	$ cd clase
	# ... %*{ Utilizar cualquier editor de texto para editar*)  test.txt
	$ ls
	test.txt
	$
	\end{lstlisting}
\end{frame}

\begin{frame}[fragile]
	\frametitle{Un buen programa de inicio}
	\begin{columns}[T] % contents are top vertically aligned
	\begin{column}[]{5cm}
		
		\begin{lstlisting}[language=Python,inputencoding=utf8, 
		basicstyle=\ttfamily\small, 
		keywordstyle=\color{keywords},
		commentstyle=\color{comments},
		stringstyle=\color{red},
		showstringspaces=false,
		identifierstyle=\color{green},caption= mi primer programa]
		print "Hola Mundo!" 
		print "Hola otra vez"
		print "Me gusta teclear."
		print "Esto es divertido."
		print 'Si! a teclear'
		print "Yo preferiria que 'no'."
		print 'Yo "dije" no tipear esto'
		\end{lstlisting}
	\end{column}
	\begin{column}[]{5cm}
		Para poder utilizar acentos y eñes en español, requerimos el encoding UTF-8
		\begin{lstlisting}[language=Python, 
		basicstyle=\ttfamily\small, 
		keywordstyle=\color{keywords},
		commentstyle=\color{comments},
		stringstyle=\color{red},
		showstringspaces=false,
		identifierstyle=\color{green},caption= preambulo en python]
		# -*- coding: utf-8 -*-
		\end{lstlisting}
		
	\end{column}
	\end{columns}
	\begin{block}{Ejercio Avanzado}
	Coloque una	\# (Almohadilla ó octothorpe) al inicio de una linea. ¿Que hace?
	\end{block}
\end{frame}	
\begin{frame}[fragile]
	\frametitle{Comentarios y el caracter de almohadilla}
	Los comentarios son muy importantes, explican en ``cristiano'' lo que quieres hacer. \alert{Regla}
			\begin{lstlisting}[language=Python,inputencoding=utf8, 
			basicstyle=\ttfamily\small, 
			keywordstyle=\color{keywords},
			commentstyle=\color{comments},
			stringstyle=\color{red},
			showstringspaces=false,
			identifierstyle=\color{green},caption= comentarios]
			# Un comentario, estos son para que 
			# puedas leer tu programa despues.
			# Todo lo que va despues de # es 
			# ignorado por Python.
			
			print "Podria tener codigo como esto" # y el comentario despues es ignorado
			
			# puedes utilizar la almohadilla para "desactivar" 
			# codigo o comentar un pedazo de codigo como:
			# print "Esto no correra"
			
			print "Esto correra"
			\end{lstlisting}		
\end{frame}

\begin{frame}[fragile]{Continuación de comentarios}
	\begin{block}{Preguntas}
		\begin{itemize}
			\item ¿Por qué \# en \textcolor{red}{print} \verb|"Hi \# there."| no es ignorado?
			\item Si \# es para comentario, ¿Como es posible que \verb|-*- coding: utf-8 -*-| funcione?
		\end{itemize}
	\end{block}
\end{frame}

\begin{frame}[shrink,fragile]
	\frametitle{Números y Matemáticas}
	\begin{lstlisting}[language=Python,inputencoding=utf8, 
	basicstyle=\ttfamily\small, 
	keywordstyle=\color{keywords},
	commentstyle=\color{comments},
	stringstyle=\color{red},
	showstringspaces=false,
	identifierstyle=\color{green},caption= Problema de granja]
	print "Tipo de problemas de una granja"
	
	print "Gallinas", 25 + 30 / 6
	print "Toros", 100 - 25 * 3 % 4
	
	print "Contemos ahora los gallos"
	
	print 3 + 2 + 1 - 5 + 4 % 2 - 1 / 4 + 6
	
	print "Es verdad que: 3 + 2 < 5 - 7?"
	
	print 3 + 2 < 5 - 7
	
	print "Cuanto es 3 + 2?", 3 + 2
	print "Cuanato es 5 - 7?", 5 - 7
	
	print "Oh,Por eso es que es falso"
	
	print "How about some more."
	
	print "Es mayor?", 5 > -2
	print "Es mayor o igual?", 5 >= -2
	print "Es menor o igual?", 5 <= -2
	\end{lstlisting}
	Cambiemos una división por \verb|7.0/4.0|
\end{frame}

\begin{frame}[plain,shrink,fragile]
	\frametitle{Variables y Nombres}
	\begin{lstlisting}[language=Python,inputencoding=utf8, 
	basicstyle=\ttfamily\small, 
	keywordstyle=\color{keywords},
	commentstyle=\color{comments},
	stringstyle=\color{red},
	showstringspaces=false,
	identifierstyle=\color{green},caption= Problema de Wheels Uniandes, basicstyle=\tiny]
	carros = 100
	espacio_en_un_carro = 4.0
	conductores = 30
	pasajeros = 90
	carros_no_manejados = carros - conductores
	carros_manejados = conductores
	capacidad_de_carpooling = carros_manejados * espacio_en_un_carro
	promedio_de_pasajeros_por_carro = pasajeros / carros_manejados
	
	
	print "Existen", carros, "carros disponibles"
	print "Tan solo hay", conductores, "conductores disponibles."
	print "Van a haber", carros_no_manejados, "carros vacios hoy"
	print "Podemos llevar", capacidad_de_carpooling, "personas"
	print "Tenemos", pasajeros, "personas para hacer carpooling"
	print "Necesitamos al menos", promedio_de_pasajeros_por_carro, "en cada carro"
	\end{lstlisting}
	
	\begin{block}{Preguntas}
		\begin{itemize}
			\item ¿Que diferencia hay entre $ = $ y \verb|==| ?
			\item ¿Que podrá hacer \verb|print "Hola %s, me llamo %s ." %("tu", "david")|
			\item ¿Por que utilizamos \verb|espacio_en_un_carro = 4.0|?
		\end{itemize}
	\end{block}
\end{frame}
\begin{frame}[plain,shrink,fragile]
	Vamos a trabajar con \alert{strings},un string es como se un programa va dar un pedazo de información a un humano
	\begin{lstlisting}[language=Python, 
	basicstyle=\ttfamily\small, 
	keywordstyle=\color{keywords},
	commentstyle=\color{comments},
	stringstyle=\color{red},
	showstringspaces=false,
	identifierstyle=\color{green},caption= Problema de Wheels Uniandes, basicstyle=\tiny]
	mi_nombre = 'David Cardozo'
	mi_edad = 19 #
	mi_altura = 170 # cms
	mi_peso = 180 # lbs
	mis_ojos = 'negros'
	mis_dientes = 'blancos'
	mi_cabello = 'cafe'
	
	print "Vamos hablar de %s." % mi_nombre
	print "El es % d centimetros de alto" % mi_altura
	print "El tiene % d libras de peso." % mi_peso
	print "Soy bien pesado"
	print "El tiene ojos de color %s  y cabello %s ." % (mis_ojos, mi_cabello)
	print "Sus dientes %s dependen del cafe." % mis_dientes
	

	print "Si yo sumo % d, % d, y % d obtengo % d." % (
	mi_edad, mi_altura, mi_peso, mi_edad + mi_altura + mi_peso)
	\end{lstlisting}
\end{frame}

\begin{frame}[plain,shrink,fragile,]
	\begin{lstlisting}[language=Python, 
	basicstyle=\ttfamily\small, 
	keywordstyle=\color{keywords},
	commentstyle=\color{comments},
	stringstyle=\color{red},
	showstringspaces=false,
	identifierstyle=\color{green},caption= Conversación con un Programador, basicstyle=\tiny]
	x = "Existen %s tipos de personas" % 10
	binario = "binario"
	no = "no"
	y = "Aquellas que saben %s y aquellos que %s." % (binario, no)

	print x
	print y

	print "Yo dije %r." % x
	print "I tambien dije: '%s'." % y

	chistoso = False
	evaluacion_chiste = "No es chistoso el chiste?! %r"

	print evaluacion_chiste % chistoso

	w = "Esta es la parte izquierda de..."
	e = "un string con esta parte a la derecha."

	print w + e
	\end{lstlisting}
	\begin{block}{importante}
		¿Que podría hacer: \verb|print "." * 10 |?
	\end{block}
\end{frame}

\begin{frame}[fragile]
	\begin{lstlisting}[language=Python, 
	basicstyle=\ttfamily\small, 
	keywordstyle=\color{keywords},
	commentstyle=\color{comments},
	stringstyle=\color{red},
	showstringspaces=false,
	identifierstyle=\color{green},caption= Dias de la semana, basicstyle=\tiny]
	
	# -*- coding: utf-8 -*-
	dias = "Lun Mar Mie Jue Vie Sab Dom"
	meses = "Ene\nFeb\nMar\nAbr\nMay\nJun\nJul\nAgo"
	
	print "Estos son los dias: ", dias
	print "Estos son los meses: ", meses
	
	print """
	Hay algo extrano aqui
	con tres doble citas.
	Vamos a ser poder meter un monton de cosas.
	Casi 4 lineas si queremos, o 5, o 6.
	"""
	\end{lstlisting}
\end{frame}




\begin{frame}[fragile]
	\begin{lstlisting}[language=Python, 
	basicstyle=\ttfamily\small, 
	keywordstyle=\color{keywords},
	commentstyle=\color{comments},
	stringstyle=\color{red},
	showstringspaces=false,
	identifierstyle=\color{green},caption= leer un archivo, basicstyle=\tiny]
	from sys import argv
	script, filename = argv
	
	txt = open(filename)
	
	print "Aqui esta su archivo %r:" % filename
	print txt.read()
	
	print "Escribirlo otra vez:"
	file_again = raw_input("> ")
	
	txt_again = open(file_again)
	
	print txt_again.read()
	
	
	\end{lstlisting}
\end{frame}

\begin{frame}[fragile]
	\begin{lstlisting}[language=Python, 
	basicstyle=\ttfamily\small, 
	keywordstyle=\color{keywords},
	commentstyle=\color{comments},
	stringstyle=\color{red},
	showstringspaces=false,
	identifierstyle=\color{green},caption= Escribiendo un texto, basicstyle=\tiny]
	from sys import argv
	script, filename = argv
	
	print "We're going to erase %r." % filename
	print "If you don't want that, hit CTRL-C (^C)."
	print "If you do want that, hit RETURN."
	
	raw_input("?")
	
	print "Opening the file..."
	target = open(filename, 'w')
	
	print "Truncating the file.  Goodbye!"
	target.truncate()
	
	print "Now I'm going to ask you for three lines."
	
	line1 = raw_input("line 1: ")
	line2 = raw_input("line 2: ")
	line3 = raw_input("line 3: ")
	
	print "I'm going to write these to the file."
	
	target.write(line1)
	target.write("\n")
	target.write(line2)
	target.write("\n")
	target.write(line3)
	target.write("\n")
	
	print "And finally, we close it."
	target.close()
	\end{lstlisting}
\end{frame}
\begin{frame}[fragile]
	\begin{lstlisting}[language=Python, 
	basicstyle=\ttfamily\small, 
	keywordstyle=\color{keywords},
	commentstyle=\color{comments},
	stringstyle=\color{red},
	showstringspaces=false,
	identifierstyle=\color{green},caption= Condicionales, basicstyle=\tiny]
	from sys import argv
	
	personas = 20
	gatos = 30
	perros = 15
	
	
	if personas < gatos:
	print "Too many cats! The world is doomed!"
	
	if personas > gatos:
	print "Not many cats! The world is saved!"
	
	if personas < perros:
	print "The world is drooled on!"
	
	if personas > perros:
	print "The world is dry!"
	
	
	perros += 5
	
	if personas >= perros:
	print "People are greater than or equal to dogs."
	
	if personas <= perros:
	print "People are less than or equal to dogs."
	
	
	if personas == perros:
	print "People are dogs."
	\end{lstlisting}
\end{frame}
\begin{frame}[fragile]
	\begin{lstlisting}[language=Python, 
	basicstyle=\ttfamily\small, 
	keywordstyle=\color{keywords},
	commentstyle=\color{comments},
	stringstyle=\color{red},
	showstringspaces=false,
	identifierstyle=\color{green},caption= 
	loops, basicstyle=\tiny]
	numero_leido = raw_input("inserta un numero >> ")
	numero = int(numero_leido)
	contador = 0
	for i in range(1,numero+1):
	if (numero% i)==0:
	contador = contador + 1
	
	if contador==2:
	print "el numero es primo"
	else print "el numero no es primo"
	\end{lstlisting}
\end{frame}

	
\end{document}