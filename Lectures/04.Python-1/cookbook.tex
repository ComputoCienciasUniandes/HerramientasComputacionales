\documentclass[12pt]{article}
\usepackage{enumerate}
\usepackage[hmargin=2.0cm,vmargin=1cm]{geometry}
\usepackage[utf8]{inputenc}
\usepackage{graphicx}
\usepackage{float}
\usepackage{cite}
\usepackage{natbib}

\title{\begin{LARGE}
{Cookbook Python}
\end{LARGE}}

\begin{document}

\maketitle


%\section{Que es python?}

%Python es un lenguaje de programaci\'on de alto nivel

En este documento encontrar\'an paso a paso lo que realizaremos en clase, 
durante el documento habr\'an ejercicios los cuales no se deber\'an de 
entregar, pero se recomienda al estudiante hacerlos para reforzar lo visto 
en clase. 

\section{Empezando}

Para abrir Python escribimos en la terminal \verb"python" lo que arrojar\'a lo siguiente:

\begin{verbatim}
python
Python 2.7.4 (default, Sep 26 2013, 03:20:26) 
[GCC 4.7.3] on linux2
Type "help", "copyright", "credits" or "license" for more information.
>>> 
\end{verbatim}

 Las 3 flechas \verb">>>" indican que estamos dentro de python y podemos empezar a escribir. Por ejemplo:

\verb">>> 1"

\verb">>> 1"\\

Es muy para ut\'il escribir aritmetica e.g:

\verb">>> 3*5" 

\verb">>> 15" \\

Si quiero dividir dos numeros tengo que especificar que son decimales de lo contrario el resultado sera aproximado al 
entero mas cercano. Para especificar que un n\'umero es un decimal utilizamos el punto as\'i:\\

\verb">>> 40/25.0"

\verb">>> 1.6" \\

otra operaci\'on importante es el operador modulo ($\%$) que se define como el sobrante de la division del numero a entre b veces.

\verb">>> 10%3"

\verb">>> 1" \\

ya que 3 puede estar 3 veces en 10 y hay un sobrante de 1.\\

Con el fin de escribir programas lo mejor ser\'a abrir un editor de texto:\\

\verb"emacs holamundo.py"\\

Ac\'a es donde escribiremos todos nuestros programas.\\

Nuestro primer programa ser\'a imprimir \verb"Hola mundo", para eso escribimos lo siguiente:\\

\verb"print 'hola mundo'"

Las comillas sencillas o dobles se pueden usar para indicar que imprimimos strings (cadenas de caracteres),
Para ejecutar el programa escribimos en la terminal:\\

\verb"python holamundo.py"\\
\verb"Hola mundo"

\section{Variables:}

Que tipo de variables existen en \verb+python+?, Para esto abramos un documento \verb"emacs variables.py & " y escribamos:

\begin{verbatim}
a = 1
b = 1.0
c = "hola"

print a, b, c
print type(a), type(b), type(c)
\end{verbatim}

Esto arrojar\'ia el siguiente resultado:

\begin{verbatim}
1 1.0 hola
<type 'int'> <type 'float'> <type 'str'>
\end{verbatim}

Las variables tipo \verb"int" guardan n\'umeros enteros, mientras que las tipo \verb"float" guardan flotantes que son numeros relaes y 
las tipo \verb"str" guardan strings (Cadenas de caracteres).

\section{Condicionales (if y while)}

Podemos controlar los programas poniendo condiciones en el. Por ejemplo si queremos imprimir algo 
solo si alguna condicion cumple, por ejemplo si $a == b$, podemos utilizar el codicional \verb+if+.

\begin{verbatim}
a = 2.0
b = 3.0
if (a==b):
   print a, "es igual que", b
elif (a>b):
   print a, "es mayor que ". b
else:
   print b, "es mayor que", a
\end{verbatim}

En este caso el resultado de este programa ser\'a:\\

\verb"3.0 es mayor que 2.0"\\

La sintaxis de \verb"if" es: entre parentesis la condici\'on (En programaci\'on == es igual, $>= $ es mayor o igual, $<=$ es menor o igual)
seguido por dos puntos (:) y luego indexado (espaciado) lo que se quiera que vaya dentro de la condicion. \verb"elif" funciona como la contraccion 
de \verb"else if". \\

Tambien podemos poner condiciones usando \verb"while"

\begin{verbatim}
i = 0
while (i<10):
   print i
   i += 1
\end{verbatim}

Esto va a producir la siguiente cuenta regresiva:

\begin{verbatim}
0
1
2
3
4
5
6
7
8
9
\end{verbatim}

{\bf Ejercicio: Escriba un programa en python que pida al usuario entrar un n\'umero entero 
y si este es impar que pare de pedir n\'umero, pero si es par que siga pidiendo n\'umeros.}

{\bf Ayuda: Para recibir n\'umero del usuario se puede usar \verb+raw_input("Escriba un n\'umero: " )+} 


\section{Iteraci\'on}

Otra forma de controlar el programa es mediante iteraciones esto lo hacemos mediante un \verb"for":\\

\begin{verbatim}
for i in range(5):
  print "hola"
\end{verbatim}

Lo que arrojara lo siguiente:

\begin{verbatim}
hola
hola
hola
hola
hola
\end{verbatim}

\section{Listas}

En muchos casos es ut\'l, crear listas con diferentes tipos de variables por ejemplo: 

\begin{verbatim}
Lista = [1.0, 3, "hola"]
\end{verbatim}

Si queremos imprimir toda la lista hacemos: 

\verb"print Lista" \\

Podemos llamar a un elemento en part\'icular de la lista solo conociendo su posici\'on, En python las posiciones empiezan desde
cero, es decir que en este caso el primer elemento de la lista $1.0$ lo podemos llamar as\'i:\\

\verb"print Lista[0]" 

Y as\'i los siguientes elementos:\\

\begin{verbatim}
Lista[1]
Lista[2] 
\end{verbatim}

imprimiran respectivamente \verb"3" y \verb"'hola'" \\

Si queremos imprimir los elementos 1 y 2 podemos usar: 

\begin{verbatim}
print Lista[0:2]
\end{verbatim}

Esta notaci\'on significa imprima  todos los elementos desde el $[0]$ hasta antes del elemento $[2]$ \\

Si queremos agregar mas elementos a la lista usamos \verb+append+ as\'i:\\

\verb+Lista.append(10)+ Esto agregara un elemento nuevo al final de la lista.\\

Para agregar elementos en cualquier posici\'on utilizamos \verb+insert+. \\

\verb+Lista.insert(2, "avion")+, donde el primer argumento entre parentesis indica la posici\'on 
en la cual queremos insertar el elemento, minetras que el segundo es lo que queremos insertar.\\

Si queremos borrar un elemento de la lista usamos \verb+pop+:\\

\verb+Lista.pop(1)+ done el argumento entre parentesis es la {\bf posici\'on} del elemento que quiero eleminar.\\

Mientras que en \verb+Lista.remove(2)+ el argumento es el {\bf elemento} que queremos eliminar. \\

Si queremos ordenar la lista en orden acendente podemos usar \verb+sort+: \\

\verb+Lista.sort()+ \\

Finalmente si queremos saber la longitud de la lista podemos usar \verb+len(Lista)+

{\bf Hacer una lista con todos los n\'umeros impares del 1 al 100} 
{\bf Explicar y dar ejemplos del uso de extend, index, reverse y count}

\section{NumPy :}
NumPy es la la librer\'ia n\'umerica de Python, dentro de esta encontraremos funciones 
trigonometricas, funciones especiales, alegra lineal (numpy.linalg) y mucho m\'as. 
.

Para usar la librer\'ia de NumPy hay que llamarla al inicio de nuestro programa, para esto la importamos 
usando \verb+import+

\verb+import numpy as np+ \\

Ahora cualquier funcion que usemos de NumPy debe de ir con el alias \verb+np.+ antes del nombre funci\'on.

\verb+>>>np.cos(np.pi)+ \\
\verb+-1.0+

{\bf Ejercicio: Investigar como funciona la libreria de algebra lineal de NumPy, Dar ejemplos de como se hace 
la multiplicacion matricial, suma/resta de matrices, como sacar determinantes y autovalores.}

\section{Arreglos}

A diferencia de las listas los arreglos nos permiten hacer operaciones matematicas. Estos hacen parte 
de NumPy, los arregos se definen as\'i:\\

\verb+x = np.array([])+, en este caso hemos creado un arreglo vacio. Si queremos crear un arreglo de zeros 
podemos usar:\\

\verb+np.zeros(5)+ Lo que arrojar\'a: 
 
\verb+array([ 0.,  0.,  0.,  0.,  0.])+\\

O si quiero hacer un arreglo de unos:\\

\verb+ones(5)+ \\

\verb+array([ 1.,  1.,  1.,  1.,  1.])+\\

Si queremos hacer una matriz de $3\times5$ podemos usar:\\

\verb+zeros([3, 5])+\\

\verb+array([[ 0.,  0.,  0.,  0.,  0.],[ 0.,  0.,  0.,  0.,  0.],[ 0.,  0.,  0.,  0.,  0.]])+

Si queremos borrar un elemento de un arreglo \verb+x+ usamos la funci\'on delete. \\

\verb+np.delete(x, 10)+ Donde el primer argumento es el arreglo y el segundo la {\bf posici\'on} que quiero 
eleminar. De manera similar funciona \verb+append+

{\bf Ejercicio: Hacer dos arreglos con numeros diferentes usando \verb+if+ y hacer operaciones matematicas entre ellos.}

Si quieres aprender mas sobre arreglos puedes ir a: \\
\verb+http://docs.scipy.org/doc/numpy/reference/arrays.html+

%section{where}

\section{N\'umeros aleatorios}

Para genera n\'umero aleatorios usamos el funci\'on de NumPy \verb+np.random+ dentro de esta funci\'on
hay varias funciones que generan numeros aleatorios segun varias distribuciones. Si queremos n\'umeros
aleatorios entre $[0, 1)$ podemos usar \verb+random+:\\

\verb+np.random.random(5)+ \\
\verb+array([ 0.85099041,  0.63817084,  0.79495685,  0.22457286,  0.68746815])+\\

{\bf Ejercicio: Hacer una matriz de 3xn donde n sea un numero cualquiera, con n\'umeros aleatorios entre 
-1 y 1}\\

{\bf Ejercicio: Ilustrar 5 ejemplos mas de funciones generadoras de n\'umeros aleaotrios dentro de np.random y explicar como se usan.}

{\bf Ejercicio: Ir a \verb+codeacademy.com+ y realizar los modulos de "Condicionales y Control de Flujo" y "Listas y Diccionarios".}
%\section{iPython, iPython notebook}

\section{Leyendo y escribiendo archivos:}

\subsection{Leyendo archivos:}

Dentro de \verb+numpy+ esta la funci\'on \verb+loadtxt+ esta nos permite leer archivos de datos. \\

\verb+data=np.loadtxt("data.txt")+\\

{\bf En \verb+https://github.com/jngaravitoc/HerramientasComputacionales/tree/master/+ \\
 \verb+Lectures/5.Python-2/data+ esta el archivo \verb+data.ascii+ leerlo y responder las 
siguientes preguntas:\\}
{\bf 1. Que tipo de variables es el archivo \verb+data.txt+ cuando es llamado por \verb+loadtxt+?}\\
{\bf 2. Con lo visto de arreglos, seleccionar las coumnas de los datos.}\\
{\bf 3. Sacar la desviacion standard y promedio de las primeras 3 columnas }

\subsection{index:}

Imaginemos que tenemos dos columnas, una con edades y la otra con alturas y queremos relacionar datos de edad y altura. Como selecionaria las edades que corresponen a alturas mayores a $1.70cm$?

\section{Funciones:}

Las funciones nos permiten estructurar los programas y organizarlos. Una funcion realize una parte 
del programa y se ejecuta cuando la llamemos.\\

{\bf Definiendo la funci\'on:}\\

\begin{verbatim}
def square(x): # entre el parentesis van los argumentos que necesitan la funci\'on.
   return x**2
\end{verbatim}

\verb+def+ es el comando que indica al programa que vamos a definir una funcion.\\
\verb+square+ es el nombre que le di a esta funcion en especifico, pero puedo utilizar el nombre que yo quiera.\\
\verb+return+ retorna el resultado de la funci\'on.\\

{\bf Llamando a mi funci\'on:}

\verb+square(10)+ que arrojar\'a como resultado \verb+100+\\

{\bf{Hacer una funci\'on que haga los siguiente cambios de cordenadas:\\
1. De Millas a Km\\
2. De Pulgadas a cm\\ 
3. De Litros a cm$^{3}$\\}}

\section{Recurrencia:}

Cuando queremos hacer un programa que haga operaciones recurrentes, podemos llamar
una funcion dentro de la misma funci\'on. Ejemplo:\\

\begin{verbatim}

def factorial(n):
    if n == 1:
        return 1
    else:
        return n * factorial(n-1)

\end{verbatim}

%\section{Matplotlib}
% Graficas, Log -log, errores, histogramas, labels, 

\section{IPython Notebook}

\subsection{Timing}

Para saber cuanto demora un proceso en el IPython notebook podemos usar el comanto \verb+%timing+

\begin{verbatim}
n = 100000
%timeit sum([1. / i**2 for i in range(1, n)])
\end{verbatim}

Esto va a arrojar:\\ 

\verb+1 loops, best of 3: 19.5 ms per loop+

\begin{verbatim}
%timeit np.sum(1. / np.arange(1., n) ** 2)
\end{verbatim}

Esto va a arrojar:\\

\verb+1000 loops, best of 3: 1.05 ms per loop+

\bibliography{references}{}
\bibliographystyle{plain}
\end{document}
