\documentclass{beamer}
\usepackage{amsmath}
\usetheme{default}
\begin{document}
%--------------------------------------------------

\begin{frame}
\begin{Huge}
\begin{center}
\textrm{UNIX \& Comandos basicos}
\end{center}

\end{Huge}
\end{frame}

%--------------------------------------------------


\begin{frame}{\textsc{UNIX}}

{\LARGE \textrm{Unix es un sistema operativo dise\~nado en los a\~nos 60's con el fin de ofrecer estabilidad en ambientes multi-usuario y multi-tarea.}}

\end{frame}

%--------------------------------------------------


\begin{frame}{\textrm{UNIX}}

\textrm{SO mas usado en el mundo:}
\begin{figure}
\includegraphics[scale=0.25]{fig2.png}
\end{figure}
{\scriptsize \textrm{Tomado de: http://www.pingdom.com}}\\
\textrm{SO mas usado por astronomos:}
\begin{figure}
\includegraphics[scale=0.2]{fig1.png}
\end{figure}
{\scriptsize \textrm{Tomado de: http://www.astrobetter.com/os-apt-astronomers/}}\\


\end{frame}

%--------------------------------------------------


\begin{frame}{\textsc{Como empezar a usar UNIX?}}

\begin{Large}
\textrm{La Terminal es el lugar en el cual se controla la maquina.
All\'i se escibre texto el cual es interpretado por la maquina como comandos.
En Ubuntu se puede abrir la terminal presionando \texttt{Ctrl +Alt + t }}
\end{Large}

\begin{figure}
\includegraphics[scale=0.3]{fig3.png}
\end{figure}

\end{frame}

%--------------------------------------------------


\begin{frame}{\textsc{Comandos basicos:}}

\begin{center}
{\large \textsc{Directorios:}}
\end{center}
\begin{itemize}
\item \texttt{pwd}: \textrm{En que directorio estoy.}\\
\item \texttt{ls}: \textrm{Muestra los directorios que hay.}\\
\item \texttt{mkdir directorio}: \textrm{Crea un directorio llamado \texttt{directorio}}. 
\item \texttt{rmdir directorio}: \textrm{Elimina el directorio llamado \texttt{directorio}}.
\end{itemize}

\end{frame}


%--------------------------------------------------


\begin{frame}{\textsc{Comandos basicos:}}

\begin{center}
{\large \textsc{Archivos:}}
\end{center}

\begin{itemize}
\item \texttt{mv archivo1 archivo2:} \textrm{Mueve} \texttt{archivo1} \textrm{a} \texttt{archivo2}.
\item \texttt{cp:archivo1 archivo2} \textrm{Copia} \texttt{archivo1} \textrm{a} \texttt{archivo2}.
\item \texttt{rm archivo:} \textrm{Borra} \texttt{archivo}.
\item \texttt{less archivo:} \textrm{Permite ver dentro de} \texttt{archivo.}
\item \texttt{head archivo:} \textrm{Muestra las primeras 10 lin\'eas de} \texttt{archivo.}
\item \texttt{tail arhcivo:} \textrm{Muestra las ultimas 10 lin\'eas de} \texttt{archivo}.
\item \texttt{cat archivo1 archivo2:} \textrm{Concatena} \texttt:{archivo1} \textrm{con} \texttt{archivo2}.
\item \texttt{grep 'palabra' archivo:} \textrm{busca} \texttt{palabra} \textrm{dentro de} \texttt{archivo}.
\item \texttt{wc archivo:} \textrm{Imprime numero de lineas, palabras y bytes en} \texttt{archivo.}
\end{itemize}
\end{frame}

%--------------------------------------------------


\begin{frame}{\textsc{Comandos basicos:}}

\begin{center}
{\large \textsc{Comandos utiles:}}
\end{center}

\begin{itemize}
\item \texttt{man command:} \textrm{Muestra el manual de} \texttt{command}.
\item \texttt{history:} \textrm{Muestra una lista de los ultimos comandos utilizados.}
\item \texttt{Ctrl+r:} \textrm{Busca en la historia de comandos.}
\item \textrm{Tab:} \textrm{Presionar Tab en la mitad de un comando o un archivo y se mostrar\'an opciones para completarlo.}
\item \textrm{du -h datos.txt:} \textrm{Muestra el tama\~no del archivo datos.txt}
\item \textrm{wget url:} \textrm{Baja archivos de la direcci\'on de internet dada.} 
\end{itemize}

\end{frame}

%--------------------------------------------------

\begin{frame}{\textsc{Comprimir y descomprimir archivos:}}

\begin{itemize}
\item \texttt{gzip archivo:} \textrm{Comprime el archivo a archivo.gz}
\item \texttt{gunzip archivo.gz:} \textrm{Descomprime archivo.gz}
\item \texttt{tar -zcvf archivo.tar archivo:} \textrm{Comprime el archivo a archivo.tar}
\item \texttt{tar -zxvf archivo.tar:} \textrm{Descomprime archivo.tar}
\end{itemize}

\end{frame}

%----------------------------------------------------


%--------------------------------------------------

\begin{frame}{\textsc{hands-on}}

\begin{center}
{\Large \textrm{ir a: }}\\

https://github.com/jngaravitoc/HerramientasComputacionales.
\end{center}

\end{frame}

%----------------------------------------------------

%--------------------------------------------------

\begin{frame}{\textsc{Referencias}}

 van Vugt, S. Beginning the Linux Command Line. 1–381 (Apress, 2009).
\end{frame}

%----------------------------------------------------

\end{document}