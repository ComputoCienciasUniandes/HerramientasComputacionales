% !TeX document-id = {0902ca31-7703-4dcf-908c-83d74402238d}
% !TeX TXS-program:compile = txs:///pdflatex/[--shell-escape]
\documentclass{beamer}
\usetheme{AnnArbor}
\usepackage[spanish]{babel}
\usepackage{lmodern}
\usepackage[T1]{fontenc}
\usepackage[utf8]{inputenc}
\usepackage{amsmath}
\usepackage{amsfonts}
\usepackage{amssymb}
\usepackage{graphicx}
\usepackage{hyperref}
\usepackage{minted}


\setbeamerfont{author in head/foot}{size={\fontsize{3.5pt}{5pt}\selectfont}}
\author{Prof. Sebastian Saaibi \& David Cardozo\inst{1}}
% - Give the names in the same order as the appear in the paper.
% - Use the \inst{?} command only if the authors have different
%   affiliation.
\title{Introducción a Git \& GitHub }
\subtitle{Control de versiones o Gestión de la configuración} % A subtitle is optional and this may be deleted
%\logo{\includegraphics[height=0.8cm]{universidaddelosandes.png}\vspace{220pt}} 
\logo{\includegraphics[height=0.8cm]{universidaddelosandesciencias.png}}
%\logo{\includegraphics[height=0.8cm]{universidaddelosandescolombia.png}
\institute[Universidad de los Andes]
{
	\inst{1}%
	Física   \\
	Lectura $6$ Herramientas Computacionales \\
	Universidad de los Andes
}
\date{\today} % - Either use conference name or its abbreviation.
\subject{PDF Information} % This is only inserted into the PDF information catalog. Can be left out. 
%\setbeamercovered{transparent}
%\setbeamertemplate{navigation symbols}{}

\begin{document}
\maketitle
\begin{frame}
	\frametitle{Temas cortos}
	\tableofcontents
\end{frame}


\section{Introducción: Control de Versiones}

\begin{frame}
	\frametitle{Control de Versiones}
	
	\textbf{Control de versiones} o \textbf{gestión de la configuración} (VCS) (SCC) (SCM)
	
	Sistemas para registrar la historia de los cambios que van sufriendo un conjunto de ficheros.
	
	 Funciona para:
	 \begin{itemize}
	 	\item Poder saber cuándo se modificó y quién lo hizo, cada fichero de un proyecto
	 	\item Poder reconstruir el estado de los ficheros al estado que tenían en algún momento del pasado.
	 	\item Poder coordinar los cambios que realiza un conjunto de desarrolladores sobre los ficheros de un proyecto.
	 	
	 \end{itemize}
	 
	 \textbf{Sistema de control de versiones (VCS)}
	 Herramienta de software que ayuda a gestiona el control de versiones: \textit{Ej: SVN, Git}
\end{frame}

\section{Control de versiones con Git}

\begin{frame}[allowframebreaks]
	\frametitle{git}
	Git es n sistema de control de versiones distribuido.
	\begin{block}{Repositorio o repo}
		Guarda la historia completa de un subconjunto de ficheros del projecto
	\end{block}
	
	\textbf{Github} ofrecen hospedaje para git en la red
	\begin{itemize}
		\item Útiles para colaborar entre varios miembros de un equipo de desarrollo.
		\item Útil para desarrolladores individuales para hacer backups del repositorio
		\item Portafolio de presencia y soporte en la web.
	\end{itemize}
\end{frame}

\begin{frame}[fragile=singleslide]
	\frametitle{git: el repositorio o repo}
	\begin{block}{El repo}
		Donde git guarda la historia completa de algún subconjunto de ficheros del proyecto
	\end{block}
	Vamos a inicializar el repo de un proyecto.
	\begin{figure}
		\begin{minted}{bash}
		mkdir testgit
		cd testgit
		git init	
		\end{minted}
		\caption{Iniciando git}
	\end{figure}
		El repositorio queda inicializado, pero vacío, en:
		
		\mint{bash}|testgit/.git/|
		
\end{frame}

\begin{frame}[fragile]
	\frametitle{git: \textit{tracked files}}
	Es el subconjunto de ficheros del proyecto que forman parte del repositorio.
	
	Sus versiones se van guardando cada vez que se compromete su estado (se hace commit)
	
	\begin{block}{add command}
		\textbf{git add '*.txt'} añade recursivamente los ficheros acabados en txt al conjunto de \textit{tracked files}
	\end{block} 
	No todos los ficheros del proyecto tienen que formar parte de los \textit{tracked files}.
	
	En \textbf{.gitignore} se especifican los ficheros que no deben añadirse.
	
	
\end{frame}

\begin{frame}[fragile]
\frametitle{.gitignore}
En la raíz del proyecto el fichero \textbf{.gitignore} especifica los ficheros que no deben añadirse al repo.
\begin{minted}{bash}
# un comentario
*.aux   # ignorar los ficheros que acaben en .aux
carpeta/ # ignorar todos los ficheros del subdir carpeta/
\end{minted}


\end{frame}

\begin{frame}[fragile]
	\frametitle{git: commit de ficheros}
	Cuando estás contento con el estado del proyecto haces commit para que se refleje en el repo el estado actual de los \textit{tracked files}
	\begin{block}{¿Qué permite commit?}
		Recuperar en el futuro el estado (foto) comprometido de los tracked files.
		\textbf{ NO ES UNA COPIA DE BACKUP DE ARCHIVOS}
	\end{block}
	Es muy importante añadir un texto que explique el estado que se compromete.
	
	\mint{bash}|git commit -m 'msg'|
	

\end{frame}

\begin{frame}[fragile]
	\frametitle{git: hands on}
	\begin{minted}[]{bash}
	git config --global core.editor 'gedit'
	mkdir gittest 
	cd gittest; touch a.txt; touch b.txt
	git init
	git add a.txt
	git status
	git b.txt
	git commit
	git status
	echo "nueva linea" > a.txt
	git status
	git diff
	git commit -a
	git status
	git log
	\end{minted}

\end{frame}

\section{Repositorios distribuidos: Github}

\begin{frame}
	\frametitle{Github: Introducción}
	Permite almacenar gratuitamente todos los repos que quieras (siempre que sean públicos)
	Al principio generas pareja clave pública/privada para autenticare en el servicio github con ssh.
	
\begin{block}{Push / Pull}
	Usando git, haces \textbf{push} de tu repo a github, creando allí una réplica. Otros desarrolladores pueden hacer \textit{push} de sus cambios y \textbf{pull} de los de otros.
\end{block}

\end{frame}

\begin{frame}[fragile]
	\frametitle{Preparación para usar github}
	\begin{itemize}
		\item Creamos una cuenta gratis en github
		\item Informamos a git del nombre de usuario y del correo para que cada commit quede identificado:
	\end{itemize}
	\begin{minted}{bash}
	git config --global user.name 'username-de-github'
	git config --global user.email 'david@correo.com'
	\end{minted}
	Creación de las parejas de claves para identificación:
	\begin{minted}{bash}
	cd ~/.ssh
	ssh-keygen -t rsa -C david@correo.com
	<Enter>
	#entre una contraseña pass123
	chmod 0600 *
	\end{minted}
\end{frame}

\begin{frame}[fragile,shrink=20]
	\frametitle{git configuración}
	Creamos un repositorio remoto en \href{https://github.com/new}{https://github.com/}
	\textbf{HerramientasComputacionales}
	
\begin{minted}{bash}
git remote add origin git@github.com:username/<Herramientas...>.git
git push origin master
\end{minted}
	
	
	
\end{frame}


	
\end{document}